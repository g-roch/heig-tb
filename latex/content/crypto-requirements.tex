\documentclass[../report]{subfiles}
\begin{document}
\chapter{Propriété cryptographique requise}
\section{Bulletin Board}

Le Bulletin Board est l'emplacement ou sont afficher l'ensemble des bulletins et informations publique des votants. 
Toutes personnes, participant ou non à l'élection doit pouvoir y avoir accès en lecture pour vérifier que la votation se passe correctement~\cite{gharadaghy_verifiability_2010}.
Dans la plupart des cas il y a besoin que ce qui y est publiée ne puisse pas y être retirée.
Cependant, on peut choisir de permettre à un votant de changer son vote (jusqu'au moment du décompte) et dans ce cas, les bulletins précédents 
peuvent aux choix être supprimer ou conserver.

\section{Authentification du votant}

Affin que seul les personnes ayant le droit de vote puissent participer, une authentification du votant doit être mise en place.
Cette authentification doit se faire sur tout les serveurs avec lesquels le votant interagis de manière confidentiel, elle ne doit
cependant pas être requise dans le cas d'une simple consultation des données publiques de la votation afin de permettre à tous de vérifier
le bon déroulement du scrutin.

Il y a 2 grande manière d'effectuer cette authentification : avec des codes d'accès unique transmis avant chaque votation ou avec des codes d'accès
conserver de votation en votation pouvant potentiellement également servir à d'autre authentification.

Lors de l'utilisation de code d'accès unique par votation, les canaux de transmission doivent être le plus sûre possible.
Dans la mesure du possible plusieurs canaux différent devrait être utilisé, et qu'un seul canal ne suffisent pas pour s'authentifié.
Un risque majeur est l'usurpation des codes d'accès et que la fraude ne puisse être détectée à temps.
Il est également important que toutes les électrices et tous les électeurs reçoivent à temps leur code d'accès, dans le cas contraire, elles 
seraient impactée dans l'exercice de leur droit de vote.
L'avantage de cette technique, c'est que les codes d'accès peuvent être généré de manière à ne permettre, dès ce moment là, aucun lien entre le vote
et l'identité du ou de la votantes.

L'utilisation de code d'accès permanent, permets de simplifié leur diffusion et les risques de retard de livraison juste avant les votations.
Un lien peu être fait par le serveur entre le bulletin de vote déposé et le numéro d'électeur ou directement l'identité du ou de la votante.
L'utilisation d'un passeport numérique comme cela se fait en Estonie~\cite{vassil_diffusion_2016} permet de simplifié considérablement le processus
pour les votants.

\todo{}
\section{Confidentialité du vote}
La confidentialité du vote est le fait que personne ne peut faire le lien entre une personne
et ce qu'elle a voté.
C'est le premier rempart contre la contrainte sur les votants.

Un confidentialité inconditionnel (c.-à-d.\ quelque soit le niveau de compromission des serveurs)
est incompatible avec la vérifiabilité universelle (même c.-à-d.) du 
résultat~\cite{chevallier-mames_incompatible_2010,gharadaghy_verifiability_2010}.
\todo{\originalcite{chevallier-mames_incompatible_2010} n'est pas accessible → vérifié les infos de l'abstract dans d'autre document}
\todo{}

\subsection{Vote sans contrainte possible}

Pour qu'il n'y a pas de contrainte possible sur le votant, ce dernier ne doit pas pouvoir prouver
ce qu'il a voté. 
Si le votant choisi ou est contraint de prouver le choix de vote qu'il a fait, cela limit
son libre choix pour le vote.
\todo{}

\section{Verifiabilité}

\begin{important}{Important}
  La vérifiabilité se base sur des propriétés cryptographiques (sauf s'il y a publication 
  de l'ensemble des votes) et donc on doit partir du principe que l'ordinateur sur lequel 
  est effectué la vérification n'est pas compromi.

  Cela vaut lors de la vérification du décompte mais également 
  lors de la vérification par le votant de son vote.
\end{important}
\todo{}

\subsection{pour le votant}
Le votant reçois immédiatement une preuve que son vote a belle et bien été pris en 
compte~\cite{adida_ballot_2006}.
\todo{ci-dessus, pas le cas chez moi}

Dans Helios~\cite{noauthor_helios_nodate}, si Alice vérifie que son bulletin contient
correctement sont vote, le système l'empèche d'utilise ce bulletin comme vote pour limiter
la coercition possible des votants.
Alice n'a ainsi pas de preuve qu'elle peut fournir à autrui de ce qu'elle a voté.
\todo{Phase 2 de vérification? \originalcite[7]{gharadaghy_verifiability_2010}}

\subsection{du résultat}
N'importe qui (même ne participant pas à la votation) peux vérifié le résultat.
\todo{}

\end{document}
