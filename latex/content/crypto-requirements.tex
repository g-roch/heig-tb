\documentclass[../report]{subfiles}
\begin{document}
\chapter{Propriété cryptographique requise}
\section{Bulletin Board}
Concept important voir necessaire~\cite{gharadaghy_verifiability_2010}

Les bulletins de vote chiffré y sont publié et ne peuvent pas y être retiré.
Tout le monde (votants, calculateur du résultat et observateur) y on accès en lecture.
\section{Authentification du votant}
\section{Confidentialité du vote}
La confidentialité du vote est le fait que personne ne peut faire le lien entre une personne
et ce qu'elle a voté.
C'est le premier rempart contre la contrainte sur les votants.

Un confidentialité inconditionnel (c.-à-d.\ quelque soit le niveau de compromission des serveurs)
est incompatible avec la vérifiabilité universelle (même c.-à-d.) du 
résultat~\cite{chevallier-mames_incompatible_2010,gharadaghy_verifiability_2010}.
\todo{\originalcite{chevallier-mames_incompatible_2010} n'est pas accessible → vérifié les infos de l'abstract dans d'autre document}

\subsection{Vote sans contrainte possible}

Pour qu'il n'y a pas de contrainte possible sur le votant, ce dernier ne doit pas pouvoir prouver
ce qu'il a voté. 
Si le votant choisi ou est contraint de prouver le choix de vote qu'il a fait, cela limit
son libre choix pour le vote.

\section{Verifiabilité}

\begin{important}{Important}
  La vérifiabilité se base sur des propriétés cryptographiques (sauf s'il y a publication 
  de l'ensemble des votes) et donc on doit partir du principe que l'ordinateur sur lequel 
  est effectué la vérification n'est pas compromi.

  Cela vaut lors de la vérification du décompte mais également 
  lors de la vérification par le votant de son vote.
\end{important}

\subsection{pour le votant}
Le votant reçois immédiatement une preuve que son vote a belle et bien été pris en 
compte~\cite{adida_ballot_2006}.

Dans Helios~\cite{noauthor_helios_nodate}, si Alice vérifie que son bulletin contient
correctement sont vote, le système l'empèche d'utilise ce bulletin comme vote pour limiter
la coercition possible des votants.
Alice n'a ainsi pas de preuve qu'elle peut fournir à autrui de ce qu'elle a voté.
\todo{Phase 2 de vérification? \originalcite[7]{gharadaghy_verifiability_2010}}

\subsection{du résultat}
N'importe qui (même ne participant pas à la votation) peux vérifié le résultat.

\end{document}
