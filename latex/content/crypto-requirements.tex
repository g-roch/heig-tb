\documentclass[../report]{subfiles}
\begin{document}
\chapter{Propriété requise}
\section{Bulletin Board}

Le Bulletin Board est l'emplacement où sont affichés l'ensemble des bulletins et informations publiques des votants. 
Toute personne, participant ou non à l'élection doit y avoir accès en lecture pour vérifier que la votation se passe correctement~\cite{gharadaghy_verifiability_2010}.
Dans la plupart des cas il est nécessaire que ce qui y est publié ne puisse pas en être retiré.
Cependant, on peut choisir de permettre à un votant de changer son vote (jusqu'au moment du décompte) et dans ce cas, les bulletins précédents 
peuvent aux choix être supprimés ou conservés.

\section{Authentification du votant}

Afin que seules les personnes ayant le droit de vote puissent participer, une authentification du votant doit être mise en place.
Cette authentification doit se faire sur tous les serveurs avec lesquels le votant interagit de manière confidentielle, elle ne doit
cependant pas être requise dans le cas d'une simple consultation des données publiques de la votation afin de permettre à tous de vérifier
le bon déroulement du scrutin.

Il y a deux grandes manières d'effectuer cette authentification : avec des codes d'accès uniques transmis avant chaque votation ou avec des codes d'accès
conservés de votation en votation pouvant potentiellement également servir à d'autres authentifications.

Lors de l'utilisation de code d'accès unique par votation, les canaux de transmission doivent être le plus sûr possible.
Dans la mesure du possible plusieurs canaux différents devrait être utilisés, et qu'au moins deux canaux différents soient nécessaires pour accomplir son authentification.
Un risque majeur est l'usurpation des codes d'accès non-détectée à temps.
Il est également important que toutes les électrices et tous les électeurs reçoivent à temps leur code d'accès, car dans le cas contraire, l'exercice de leur droit de vote serait compromis.
L'avantage de cette technique, est que les codes d'accès peuvent être générés de manière à ne permettre, dès ce moment là, aucun lien entre le vote
et l'identité du ou de la votant-e.

L'utilisation de code d'accès permanents, permet de simplifier leur diffusion et d'éviter les risques de retard de livraison juste avant les votations.
Un lien peut être fait par le serveur entre le bulletin de vote déposé et le numéro d'électeur ou directement l'identité du ou de la votant-e.
L'utilisation d'un passeport numérique comme cela se fait en Estonie~\cite{vassil_diffusion_2016} permet de simplifier considérablement le processus
pour les votants.

Dans le cadre de ce projet l'authentification du votant n'a pas été faite, cependant comme le protocole mis en place en avait besoin, le POC utilise 
simplement le numéro d'électeur mentionné par le client, sans authentification.

\section{Confidentialité du vote}

La confidentialité du vote est le principe qui veut qu'aucun lien ne peut être fait entre une personne et ce qu'elle a voté.
C'est le premier rempart contre la coercition des votants.
Un vote public permet de menacer une personne pour la forcé à changer son vote, et la menace peut être mise à exécution
car le changement ou non-changement du vote sera connu de tous.

La confidentialité inconditionnelle est qu'en tout temps, le secret du vote soit préservé, 
y compris lorsque les serveurs centraux sont compromis et ce quelque soit le moment où leur compromission a lieu.
La vérifiabilité universelle est que le résultat du vote peut être vérifié quelque soit le niveau de compromission
des serveurs.

Un confidentialité inconditionnelle est incompatible avec la vérifiabilité universelle du 
résultat~\cite{chevallier-mames_incompatible_2010,gharadaghy_verifiability_2010}.
Un choix doit donc être fait sur ce qui est le plus important pour le cas d'utilisation désiré.

\subsection{Vote sans contrainte possible}

Si un votant peut prouver à des tiers ce qu'il a voté, ces dernier peuvent le contraindre à leur révéler son vote.
Dans ce cas, comme lorsque le vote est public, le votant peut être contraint dans son vote. 
Pour éviter ce problème, les scrutins traditionnels utilisent des isoloirs et des bulletins de vote 
standardisés.\footnote{Cependant avec les technologies actuelles, cette sécurité est moindre, car un votant pourrait 
	choisir (ou être contraint) de se filmer lors de son vote et ainsi en créer une preuve.}

Lors d'un vote électronique, le votant peut avoir accès à des informations qui lui permette de déchiffrer son vote.
Avec l'utilisation de machine à voter, ces informations peuvent être plus ou moins masquées au votant, mais lors
de l'utilisation de l'e-voting à travers Internet, c'est le terminal de l'utilisateur qui est utilisé.
Grâce à la maîtrise de ce terminal (en étant le propriétaire ou en l'ayant compromis) les informations permettant
de prouver le vote peuvent être récupérées et transmises.

\section{Vérifiabilité}

\begin{important}{Important}
  La vérifiabilité se base sur des propriétés cryptographiques (sauf s'il y a publication 
  de l'ensemble des votes) et donc on doit partir du principe que l'ordinateur sur lequel 
  est effectué la vérification n'est pas compromis.

  Cela vaut lors de la vérification du décompte mais également 
  lors de la vérification de son vote par le votant.
\end{important}

Le votant reçoit immédiatement une preuve que son vote a bel et bien été pris en 
compte~\cite{adida_ballot_2006}, cependant dans certain cas, le votant doit attendre que 
le scrutin soit clôt pour pouvoir faire une telle vérification.

Dans Helios~\cite{noauthor_helios_nodate}, si Alice vérifie que son bulletin contient
correctement son vote, le système l'empèche d'utiliser ce bulletin comme vote pour limiter
la coercition possible des votants.
Alice n'a ainsi pas de preuve qu'elle peut fournir à autrui de ce qu'elle a voté.

N'importe qui (même ne participant pas à la votation) peut vérifier le résultat final du scrutin.


\end{document}
