
\documentclass[../report]{subfiles}
\begin{document}
\part{Implémentation}

\chapter{Cas d'utilisation}\label{chap:usecase}

De manière générale l'utilisation du vote électronique par rapport à un vote traditionnel ouvre la voie à 
des attaques qui étaient jusqu'à présent impossible. 

Lorsque l'action de voter ne se fait plus dans un isoloir dans un bureau de vote (ce qui est déjà le cas
pour le vote par correspondance suisse) il y a une plus grande possibilité de coercition des votants.
On ne pourra jamais exclure que des votants aient été contraints dans leur choix. 
Le vote électronique ne résout pas ce problème.

\section{Cas d'utilisation}
\subsection{Vote lors d'une assemblée}

Lors d'un assemblée d'une association ou d'une entreprise/coopérative, les décisions sont souvent prises par
des votations sur des propositions.
C'est votations peuvent être faites par e-voting dans la majorité des cas, cela permet de limiter les erreurs 
de comptage et d'avoir un résultat plus rapide s'il y a beaucoup de monde.
Le vote électronique permet également d'avoir une plus grande confiance dans les résultats, car on ne fait pas 
confiance à un petit groupe chargé du décompte des voix (groupe qui ne contient pas toujours de personne de 
chaque position possible).

Le corruption des personnes chargées du décompte peut être facile, alors que si l'application utilisée pour
l'e-voting est externe au groupe, il peut être beaucoup plus difficile de la corrompre.

\begin{important}{Vote d'importance}
	Même pour de petites structures, il peut avoir des risques importants à utiliser un système de vote électronique, 
	notamment s'il y a un ou des opposants puissants (économiquement, politiquement ou informatiquement), et ce que
	ce soit par rapport à un vote spécifique ou de manière globale à la structure.
\end{important}

\subsection{Prise de décision dans un groupe d'amis}

Dans un groupe d'amis, on peut vouloir prendre une décision importante en prenant au mieux l'avis de tous. 
Bien sûr dans ce cas, le mieux ce serait qu'il y ait consensus.
Un scrutin par e-voting permet de récolter les positions de tous, sans qu'il y ait de pression à choisir l'une où 
l'autre option, car le vote est à peu près secret. 
Cette façon de faire permet également aux amis qui ne seraient pas sur place au moment du choix de donner également leur avis.
Lorsque le groupe est petit, le secret du vote est relatif, car pour lever le secret il suffit de connaître l'ensemble des
vote des autres participant et le résultat (qui est connu de tous).
Cette technique permet également d'avoir une estimation du vote d'un sous-groupe, en connaissant l'ensemble des autres votes.
Quel que soit le scrutin, du moment que les résultats détaillés sont connu de tous, il y aura toujours cette limitation du secret, et ce  
quelle que soit la méthode utilisé (e-voting versus traditionnel et type de scrutin).

%\section{Sondage rapide et anonyme au sein d'un groupe}
%
%Lorsqu'on fait des sondages, il peut être difficile de garantir l'anonymat des réponses. 
%S'il y a besoin d'anonymat et de rapidité, l'utilisation d'une solution de e-voting peut permettre d'organiser 
%de tel sondage. 



\chapter{Le protocole}

Le choix du scrutin s'est porté sur la méthode Borda (\aaref{sec:scrutin:borda}).
La publication «~\emph{A Smart Contract System for Decentralized Borda Count Voting}~»~\cite{panja_smart_2020}
propose un protocole en 2 tours mettant en œuvre la méthode Borda en garantissant la
confidentialité du vote tout le long du processus de vote.
Leur proposition se base sur un groupe cyclique multiplicatif DSA. 
L'implémentation réalisée durant ce travail se base sur leur travail en l'adaptant sur les courbes elliptiques.
Le groupe utilisé est le groupe de courbe elliptique d'ordre premier ristretto255 construit au dessus de Curve25519.
Cette courbe respecte donc l'indication que le groupe utilisé doit être d'ordre premier.
Le code source du Proof-of-concept est disponible à l'adresse \url{https://github.com/g-roch/heig-tb-report/tree/main/app},
l'ensemble de cette implémentation a été réalisé en Rust durant ce travail de bachelor.

\begin{nota}{Vote partiel}
	L'implémentation proposée et décrite dans ce document ne permet pas au votant d'effectuer des votes 
	partiels (où seule une partie des options serait classée).
\end{nota}


\section{Protocole à 2 tours}

Pour permettre la confidentialité du vote y compris envers le serveur central réceptionnant
le bulletin de vote, le protocole de vote fonctionne en 2 tours.
Lors du premier tour chaque participant publie une clé publique éphémère par candidat/option et une 
preuve NIZK\footnote{Non-interactive zero-knoledge} qu'il a bien connaissance de la clé privée.
Au début du second tour, chaque participant vérifie toutes les preuves du premier tour.
Le bulletin de vote est ensuite formé à partir de l'ensemble des clés publiques des autres 
votants et du choix de vote effectué.
Avec le bulletin de vote, chaque participant publie une preuve NIZK que son vote est bien 
une permutation des choix possible.

Soit 

\begin{itemize}
	\item $n$ le nombre de votants
	\item $k$ le nombre d'options à classer
	\item $a_j$ le score associé au rang $j$, lorsque le candidat classé en dernier reçoit un score de 0 et le suivant de 1 et ainsi de suite, $a_j = j - 1$, cependant une autre répartition des scores est possible.
	\item $v_{ij}$ est le score donné au candidat $j$ par le votant $i$.
\end{itemize}

\subsection{Premier tour}

Lors du premier tour chaque participant ($i$) génére une paire de clés sur la courbe par option possible.
La clé privée du votant $i$ pour le candidat $j$ est notée $x_{ij}$ et la clé publique associée 
est $X_{ij} = x_{ij} G$ avec $G$ le générateur du groupe ristretto255.
La clé privée est tirée uniformément dans le groupe.

Chaque participant publie les clés publiques pour chaque option $(X_{i1}, X_{i2}, \dots, X_{ik})$ ainsi que les preuves 
«~non-interactive zero-knowledge~» (NIZK) qu'ils connaissent les clés privées associées \\ $(P_K\{x_{i1}\}, P_K\{x_{i2}\}, \dots, P_K\{x_{ik}\})$. 
Avec $P_K\{x_{ij}\} = P_K\{x_{ij}: (X_{ij} = x_{ij} G)\}$.

\subsection{Second tour}

Pour le second tour chaque participant télécharge l'ensemble des données publiées au premier 
tour (voir \aaref{sec:res:data-size} pour un calcul du volume de données) et vérifie la
preuve associée à chaque clé publique.

Chaque participant $i$ calcul $Y_{ij}$ et $Z_{ij}$ pour chaque option.
\[
	Y_{ij} = y_{ij} G = \sum_{l=1}^{i-1} X_{lj} - \sum_{l=i+1}^{n} X_{lj}
\]

\[
	Z_{ij} = x_{ij} Y_{ij} + v_{ij} G
\]

Il calcule également $k$ preuves NIZK ($\Pi_m$) que le score $a_{m}$ est bien présent dans un des bulletins 
$(Z_{i1}, Z_{i2}, \dots, Z_{ik})$ pour tout $m \in (1, 2, \dots k)$.
Comme le nombre de bulletins et le même que le nombre d'options disponibles, on prouve ainsi
que le votant a bien fait une permutation des scores possibles.

\[
\Pi_m = P_K\{x_{ij}: \vee_{l=1}^{k} ((X_{il} = x{il} G) \wedge (Z_{il} - a_{m}G = x_{il} Y_{il}) )\}
\]

Chaque votant publie ses bulletins de vote $(Z_{i1}, \dots Z_{ik})$ ainsi que les $k$ preuves NIZK $(\Pi_1, \dots, \Pi_k)$.


\section{Calcul du résultat}\label{sec:res:proto:resultat}

Pour calculer ou vérifier le résultat, chaque participant doit télécharger l'ensemble des données générées lors des 
deux tours (voir \aaref{sec:res:data-size} pour un calcul du volume de données).

On commence par calculer $R_j$ pour chaque option.

\[
	R_j = \sum_{i=1}^{n} Z_{ij} = \sum_{i=1}^{n} x_{ij}Y_{ij} + v_{ij}G = G \cdot \sum_{i=1}^{n} v_{ij}
\]

Le résultat final est $r_j$ qui est donné par la relation $R_j = r_j G$.
Le calcul du résultat final est le calcul du logarithme discret sur $R_j$.
C'est un problème difficile sur de grands nombres, cependant pour des nombres 
relativement petits, une recherche par force brute est possible.
Le temps de calcul est parfaitement raisonnable et fonctionnel, même pour de très grands groupes de personnes; 
une analyse détaillée est faite dans la \aaref{sec:res:perf:resultat}.

Si le résultat est déjà connu et que seule la vérification est nécessaire, le calcul de $R_j$ est nécessaire, mais il n'y
pas de force brute à faire, car il suffit de vérifier la relation $R_j = r_j G$.

Le temps de calcul du résultat est estimé dans la \aaref{sec:res:perf:resultat}.

\end{document}
