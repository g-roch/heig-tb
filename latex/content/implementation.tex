
\documentclass[../report]{subfiles}
\begin{document}
\part{Implémentation}

\chapter{Cas d'utilisations}\label{chap:usecase}

De manière générale l'utilisation du vote électronique par rapport à un vote traditionnel ouvre la voie à 
des attaques qui était jusqu'à présent impossible. 

Lorsque l'action de voter ne se fait plus dans un isoloirs dans un bureau de vote (ce qui est déjà le cas
pour le vote par correspondance suisse) il y a une plus grande possibilité de coercition des votants.
On ne pourra jamais exclure que des votants aient été contraint dans leur choix. 
Le vote électronique ne résous pas ce problème.

\section{Cas d'utilisations}
\subsection{Vote lors d'une assemblée}

Lors d'un assemblée d'une association ou d'une entreprise/coopérative, les décisions sont souvent prise avec
des votations sur des propositions.
C'est votations peuvent être faites par e-voting dans la majorité des cas, cela permet de limité les erreurs 
de comptage et d'avoir un résultat plus rapide s'il y a beaucoup de monde.
Le vote électronique permet également d'avoir une plus grande confiance dans les résultats, car on ne fait pas 
confiance à un petit groupe chargé du décompte des voix (groupe qui ne contient pas toujours de personne de 
chaque position possible).

Le corruption des personnes chargées du décompte peux être facile, alors que si l'application utilisé pour
l'e-voting est externe au groupe, il peut être beaucoup plus difficile de les corrompre.

\begin{important}{Vote d'importance}
	Même pour de petite structure, il peut avoir des risques importants à utilisé un système de vote électronique, 
	notament s'il y a un ou des oppansants puissants (économiquement, politiquement ou informatiquement), et ce que
	se soit par rapport à un vote spécifique ou de manière globale à la structure.
\end{important}

\subsection{Prise de décision dans un groupe d'amis}




\todo{Autre cas d'utilisation?}
%\section{Sondage rapide et anonyme au sein d'un groupe}
%
%Lorsqu'on fait des sondages, il peut être difficile de garantir l'anonymat des réponses. 
%S'il y a besoin d'anonymat et de rapidité, l'utilisation d'une solution de e-voting peut permettre d'organiser 
%de tel sondage. 



\chapter{Le protocole}

\section{Protocole à 2 tours}

Pour permettre la confidentialité du vote y compris envers le serveur central réceptionnant
le bulletin de vote, le protocole de vote fonctionne en 2 tours.
Lors du premier tours chaque participant publie une clé publique par candidat/option et une 
preuve NIZK\footnote{Non-interactive zero-knoledge} que ces clés sont bien formée.
Au début du second tours, chaque participant vérifie toutes les preuves que les clés publiques
sont bien formée.
Le bulletin de vote est ensuite formé à partir de l'ensemble des clés publiques des autres 
votants et du choix de vote effectué.
Avec le bulletin de vote, chaque participant publie une preuve NIZK que son vote est bien 
une permutation des choix possible.

\section{Calcul du résultat}\label{sec:res:proto:resultat}

\todo{Est-ce log discret sur une courbe éliptique? est-ce que je dis la somme ou le produit? }
Le calcul du résultat final est le calcul du logarithme discret sur le produit de tout les 
bulletins de vote.
Ce calcul est un problème difficile sur de grand nombre, cepandant pour des nombres 
relativements petits, une recherche par brute force est possible.
Le temps de calcul est parfaitement raisonable et fonctionnel, même pour de très grand groupe, 
une analyse détaillée est fait dans la \aaref{sec:res:perf:resultat}.


\end{document}
