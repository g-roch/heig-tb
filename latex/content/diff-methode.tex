\documentclass[../report]{subfiles}
\begin{document}


  \part{Analyse des différents type de scrutins}
  
  \chapter{Critère de comparaison}
  
  Le choix des critères de comparaison des différente méthode de vote, est choix 
  éminament politique.
  C'est équivalement à répondre à la question quel candidat souhaite-t-on élire en fonction
  des votes des électeurs, cependant on veillera à ne pas prendre en compte les positions 
  politiques des différents candidat.
  On veux savoir quel candidat devrait être élus, pas comment faire pour que notre candidat 
  favori l'emporte face aux autres candidats.
  
  \section{Indépendant à la présence de «petit» candidat}
  
  La présence ou l'absence de candidat n'ayant acune chance de gagner, ne devrait pas influancer
  le résultat du scrutin.
  
  \subsection{Example}
  \newcommand\scrutinname{« Vote 1 »}
  \inputMake{_dyn/scrutin/vote1.votes.tex}
  \inputMake{_dyn/scrutin/vote1.maj1.tex}
  \inputMake{_dyn/scrutin/vote1.maj2.tex}
  %\inputMake{_dyn/scrutin/vote1.all.tex}

  \section{Paradoxe de Condorcet}
  \section{¿ Paradoxe de Borda ?}
  \section{¿ Paradoxe d'Arrow ?}
  \section{quasi unanimité du vote}
  
  \chapter{Les scrutins}

  \section{Scrutin proportionnel plurinominal}
  
  \subsection{Fonctionnement}
  
  Les citoyens votent pour une liste (de candidats) de leur choix.
  Chaque liste remporte un nombre de siège proportionnel au nombre de voix reçue.
  \cite{noauthor_scrutin_2021}
  
  \subsection{Intégrale ou avec seuil}
  \todo{définir: quotient électoral}
  En proportionnelle intégrale, chaque liste atteignant le quotient électoral, obtient
  le droit à un/des sièges.
  Un seuil de nombre de voix peux être défini pour qu'une liste ait droit à un siège.
  
  \subsection{Préférentiel ou listes bloquées}
  En listes bloquées, les sièges sont répartis au sein des listes par ordre d'apparision 
  des candidats.
  En préférentiel, les sièges sont répartis en fonction des préférences des électeurs.
  En changeant l'ordre des candidats ou en biffant/rajoutant certain ou en cumulant certain 
  candidat de multiple foix.
  
  \subsection{Répartition des sièges}
  Le siège ne pouvant pas être divisé, une méthode doit être définit pour répartir 
  correctement ces fragments de siège.
  
  \section{Scrutin uninominal majoritaire}
  
  Chaque votant vote pour son candidat/option préférée, et l'option/candidat avec le plus de
  voix l'emporte.
  
  \subsection{Majorité relative ou absolue}
  \subsubsection{Majorité absolue}
  Dans le cas d'un scrutin à majorité absolue, l'option avec plus de 50\% de vote est choisie.
  Si on exclues le cas d'un ex-equo et que l'on ne prends pas en compte les vote 
  blancs/nuls/abstansions, pour 2 options possibles, il y a forcement un gagnant.
  
  Dans le cas où aucune options n'atteint les 50\%, une tour supplémentaire est généralement
  organiser.
  Ce tour supplémentaire peux être à la majorité relative dans une volonté de limité le nombre
  de tour.
  
  \begin{nota}{Pourquoi 50\%~?}
    Pour certain scrutin plus important, le seuils est plus élevé que 50\%, example 60\% ou 70\%. 
    En quoi est-ce que 50\% est un bon choix, est-ce que ça veux dire que le choix de la moitié
    des votants n'as pas importance~? 
  \end{nota}
  
  \subsubsection{Majorité relative}
  Dans le cas d'un scrution à majorité relative, l'option avec le plus de voix est choisie, 
  quelque soit son pourcentage de voix.
  
  \subsection{Nombre de tour}
  
  Dans la pluspart des cas, il y a plusieurs tours d'organiser, soit jusqu'à ce qu'il y a 
  un candidat qui remporte une majorité absolue, soit un nombre prédéfini de tour (souvant 2).
  Habituellement, entre chaque tour 1 ou plusieurs candidats sont éliminés.
  
  \begin{nota}{Qui éliminer ?}
    Habituellement, les candidats/options avec le moins de voix sont éliminé, mais
    est-ce vraiment leurs électeurs qui sont le plus suceptible de changer d'avis ?
    Est-ce qu'un électeur d'un des 2 candidat proche mutuellement n'aurait pas plus
    de facilité à changer son vote ?
  \end{nota}

  \subsection{Algorithme}

  \begin{algorithm}[H]
  \caption{Scrutin majoritaire uninominal à 1 tour}
  \label{scrutin:maj-uni-1t}
  \begin{algorithmic}[1]
  \REQUIRE $votes$[n° du votant] = n° du choix favori
  \ENSURE Liste des n° des choix vainqueurs
  \FORALL{$vote \leftarrow votes$}
  \STATE $total[vote]$++
  \ENDFOR
  \RETURN Les indexes de max($total$)
  \end{algorithmic}
  \end{algorithm}
  
  \begin{algorithm}[H]
  \caption{Scrutin majoritaire uninominal à 2 tour}
  \label{scrutin:maj-uni-2t}
  \begin{algorithmic}[1]
  \REQUIRE $votes$[n° du votant] = Liste des choix par ordre de préférence décroisante
  \ENSURE Liste des n° des choix vainqueurs
  \STATE $majorité \leftarrow \frac{\text{len}(votes)}{2} $
  \FORALL{$vote \leftarrow votes$}
    \STATE $choix \leftarrow vote[0]$ \COMMENT{Premier choix du votant}
    \STATE $total[choix]$++
  \ENDFOR
  \IF{$\text{max}(total) > majorité$}
    \RETURN Index de max($total$)
  \ENDIF
  \STATE
  \STATE $restant \leftarrow $ index des 2 valeurs maximal de $total$
  \STATE vider $total$
  \FORALL{$vote \leftarrow votes$}
    \STATE $choix \leftarrow $ première valeur de $vote$ se trouvant dans $restant$
    \STATE $total[choix]++$
  \ENDFOR
  \IF{$\text{max}(total) > majorité$}
    \RETURN Index de max($total$)
  \ELSE
    \STATE \COMMENT{Les 2 choix restants sont ex-aequo}
    \RETURN $restant$
  \ENDIF
  \end{algorithmic}
  \end{algorithm}

  \subsection{Cas limites}

  \begin{table}[H]
    \begin{center}
      \caption{Cas limites d'un scrutin à la majorité en 1 tour}
      \label{fig:diff:maj1:caslim1}
      \adjustbox{valign=t}{
        \begin{subtable}[h]{0.45\textwidth}
          \centering
          \caption{Préférences des électeurs}
          \inputMake{_dyn/scrutin/maj1-caslim-1.votes.tex}
        \end{subtable}
      }
      \adjustbox{valign=t}{
        \begin{subtable}[h]{0.45\textwidth}
          \centering
          \caption{Résultats du scrutin}
          \inputMake{_dyn/scrutin/maj1-caslim-1.maj1.tex}
        \end{subtable}
      }
    \end{center}
  \end{table}

  On peux constater plus de 60\% des électeurs (ii et iii) préférent B ou C à A. 
  Cependant comme A a fait plus de voix que B ou C, c'est donc cette option qui est élue.
  Si B ou C s'était retirer avant le scrutin ou s'il y avait eu une candidature commune, A aurait été largement battue. 


  \section{Méthode Borda}

  \subsection{Justification choix possible}

  Dans le cas où un vote partielle est autorisé/possible (c.-à-d. certaines options sont 
  non-classées).
  Cela pourrait être bien de ne pas scoré les options (de la meilleures au pire)
  avec les valeur ($n$, $n-1$, ..., $1$) et les options non-classée à $0$.
  C'est mieux de les scorés du pire au meilleure ($1$, $2$, ..., $j$) avec $j$ le nombre
  d'option classée et laisser à $0$ les options non-classée.
  Ce choix évite de perdre les avantages de la méthode Borda si un grande partie des votants
  ne classe que 1 ou 2 candidats.
  \cite{emerson_original_2013}
  
  \subsection{Algorithme}

  \begin{algorithm}[H]
  \caption{Scrutin avec la méthode Borda (si que 2/5 classées, les points attribués sont 1 et 2)}
  \label{scrutin:borda-max-nbclassee}
  \begin{algorithmic}[1]
  \REQUIRE $votes$[n° du votant] = Liste des choix par ordre de préférence décroisante
  \ENSURE Liste des n° des choix vainqueurs
  \FORALL{$vote \leftarrow votes$}
    \STATE $point \leftarrow \text{len}(vote)$
    \FORALL{$option \leftarrow vote$}
      \STATE $total[option] \leftarrow total[option] + point$
      \STATE $point--$
    \ENDFOR
  \ENDFOR
  \RETURN Les indexes de max($total$)
  \end{algorithmic}
  \end{algorithm}
  
  \begin{algorithm}[H]
  \caption{Scrutin avec la méthode Borda (si que 2/5 classées, les points attribués sont 5 et 4)}
  \label{scrutin:borda-max-nboption}
  \begin{algorithmic}[1]
  \REQUIRE $votes$[n° du votant] = Liste des choix par ordre de préférence décroisante
  \ENSURE Liste des n° des choix vainqueurs
  \FORALL{$vote \leftarrow votes$}
    \STATE $point \leftarrow $ Nombre d'option disponible
    \FORALL{$option \leftarrow vote$}
      \STATE $total[option] \leftarrow total[option] + point$
      \STATE $point--$
    \ENDFOR
  \ENDFOR
  \RETURN Les indexes de max($total$)
  \end{algorithmic}
  \end{algorithm}
  
  \subsection{Cas limites}
  \begin{table}[H]
    \begin{center}
      \caption{Cas limites d'un scrutin utilisant la méthode Borda}
      \label{fig:diff:borda:caslim1}
      \adjustbox{valign=t}{
        \begin{subtable}[h]{0.45\textwidth}
          \centering
          \caption{Préférences des électeurs}
          \inputMake{_dyn/scrutin/borda-caslim-1A.votes.tex}
        \end{subtable}
      }
      \adjustbox{valign=t}{
        \begin{subtable}[h]{0.45\textwidth}
          \centering
          \caption{Résultat}
          \inputMake{_dyn/scrutin/borda-caslim-1A.bordatot.tex}
        \end{subtable}
      }\\[1em]
      \adjustbox{valign=t}{
        \begin{subtable}[h]{0.45\textwidth}
          \centering
          \caption{Préférences des électeurs}
          \inputMake{_dyn/scrutin/borda-caslim-1B.votes.tex}
        \end{subtable}
      }
      \adjustbox{valign=t}{
        \begin{subtable}[h]{0.45\textwidth}
          \centering
          \caption{Résultat}
          \inputMake{_dyn/scrutin/borda-caslim-1B.bordatot.tex}
        \end{subtable}
      }
    \end{center}
  \end{table}

  \begin{table}[H]
    \begin{center}
      \caption{Cas limites d'un scrutin utilisant la méthode Borda}
      \label{fig:diff:borda:caslim2}
      \adjustbox{valign=t}{
        \begin{subtable}[h]{0.45\textwidth}
          \centering
          \caption{Préférences des électeurs}
          \inputMake{_dyn/scrutin/borda-caslim-2.votes.tex}
        \end{subtable}
      }\\[1em]
      \adjustbox{valign=t}{
        \begin{subtable}[h]{0.45\textwidth}
          \centering
          \caption{Nombre de points disponible en fonction du nombre de candidat classé}
          \inputMake{_dyn/scrutin/borda-caslim-2.bordaclasse.tex}
        \end{subtable}
      }
      \adjustbox{valign=t}{
        \begin{subtable}[h]{0.45\textwidth}
          \centering
          \caption{Nombre de points disponible en fonction du nombre de candidat disponible}
          \inputMake{_dyn/scrutin/borda-caslim-2.bordatot.tex}
        \end{subtable}
      }
    \end{center}
  \end{table}

  \section{Méthode de Condorcet}
  \todo{}
  \subsection{Cas limites}
  \inputMake{_dyn/scrutin/condorcet-caslim-1.votes.tex}
  \inputMake{_dyn/scrutin/condorcet-caslim-1.condorcet.tex}
  \section{Vote par abbrobation}
  \todo{}
  
  \subsection{Cas limites}
  \inputMake{_dyn/scrutin/apro-caslim-1.votes.tex}
  \inputMake{_dyn/scrutin/apro-caslim-1.apro.tex}


  C gagne alors que 75\% de l'électorat lui préfère largement A et 75\% lui préfère B.

  \section{Jugement majoritaire}
  \todo{}
  
  \begin{algorithm}[H]
  \caption{Scrutin au jugement majoritaire}
  \label{scrutin:jugement-maj}
  \begin{algorithmic}[1]
  \REQUIRE $votes$[n° du votant][n° de l'option] = Le jugement de l'option (5 = parfait; 0 = À rejeter)
  \ENSURE Liste des n° des choix vainqueurs
  \STATE $majorité \leftarrow \frac{\text{len}(votes)}{2}$
  \FORALL{$vote \leftarrow votes$}
    \FORALL{$option \leftarrow vote$}
      \STATE $jugement \leftarrow vote[option]$
      \STATE $total[option][jugement] ++$
    \ENDFOR
  \ENDFOR
  % TODO
  \end{algorithmic}
  \end{algorithm}

  \subsection{Cas limites}
  \inputMake{_dyn/scrutin/jugmaj-caslim-1A.votes.tex}

  \inputMake{_dyn/scrutin/jugmaj-caslim-1A.jugmaj.tex}

  \inputMake{_dyn/scrutin/jugmaj-caslim-1B.votes.tex}

  \inputMake{_dyn/scrutin/jugmaj-caslim-1B.jugmaj.tex}
  
  \section{Méthode de Coombs}
  \todo{}
  \section{Vote alternatif}
  Aussi appelé vote préférencielle transférable ou Méthode de R
  \todo{}
  \section{Vote cumulatif}
  \todo{}


  \chapter{Problème et avantage}
  
  \todo{Prendre en compte ou non les votes blancs/nuls et les abstentions}
  \todo{$50 \% + 1$ versus $ > 50\%$}
  \todo{Avantage: il y a de la réflexion entre les tours lors de scrutin 
  multi-tours. Tout le monde ne reste pas forcement sur ses positions}
  \todo{Vote utiles: Quel utilité~?}
  \todo{Manipulabilité du résultat (sans cassé le protocole)}

  
  %\begin{nota}[by me]{Ma note}\lipsum[1][1-5]\end{nota}
  %\begin{question}[by me]{Ma question}\lipsum[2][1-5]\end{question}
  %\begin{important}[by me]{Ma not imp!}\lipsum[3][1-5]\end{important}
  %\begin{warning}[by me]{My warn}\lipsum[4][1-5]\end{warning}

\end{document}
