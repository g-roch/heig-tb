\documentclass[../report]{subfiles}
\begin{document}

\newcommand\tabcritere[8]{
  \begin{center}
    \begin{tabular}{r|c|p{20em}}
      \hline
      Critère & Évaluation & Raison \\
      \hline
      \hline
      Avis des perdants & #1 & #2 \\
      \hline
      Indépendance aux petits candidats & #3 & #4 \\
      \hline
      Quasi unanimité du vote & #5 & #6 \\
      \hline
      Manipulabilité & #7 & #8 \\
      \hline
    \end{tabular}
  \end{center}
}


  \chapter{Introduction}

  \todo{Type de méthode de scrutin analyser}
  \todo{candidat vs options}
  \todo{Vote blanc/nuls/abstentions dans les simulations}
  \todo{Exemple fictif}
  \todo{referendhum cas particulier d'une élection}

  \section{Définition et abréviation}
  
  \begin{description}
    \item[vote utile] \aref{diff:comp:util:util}
    \item[quorum]
  \end{description}

  \part{Analyse des différents type de scrutins}

  \chapter{Critère de comparaison}
  
  Le choix des critères de comparaison des différente méthode de vote, est choix 
  éminament politique.
  C'est équivalement à répondre à la question quel candidat souhaite-t-on élire en fonction
  des votes des électeurs, cependant on veillera à ne pas prendre en compte les positions 
  politiques des différents candidat.
  On veux savoir quel candidat devrait être élus, pas comment faire pour que notre candidat 
  favori l'emporte face aux autres candidats.

  \section{Avis des perdants}

  Avec certaine méthode, un candidats peux être élus alors que 49\% du corps électoral est
  absolument contre et ce même lors qu'un autre candidat avec un score plus faible n'a aucune
  électeur contre.
  Un système de choix performent devrais prendre en compte le vote qui ne sont pas pour le candidat
  élu, afin de trouvé le candidat qui serait satisfaisant pour le plus de monde possible.

  \todo{example}
  
  \section{Indépendance à la présence de candidat perdant}

  On peut souhaiter que si une option non-élue est retirée du scrutin, l'option vainqueure reste
  la même.
  Cependant même si ce critère n'est pas totalement respecté pour des candidats avec de très bon 
  score, le retrait d'un petit candidat (avec un score faible) ne doit pas faire changer 
  le vainqueur.

  \subsection{Théorème d'impossibilité d'Arrow}

  Ce théorème nous dis que dans un scrutin où les électeurs indique leur préférences en classant les
  options les unes par repport aux autres.
  Dans ce cadre il n'existe pas de processus de choix indiscutable permettant un choix 
  cohérant dans le sens où le retrait d'un candidat non-élus, ne change pas le nom de l'élu.

  Dans les types de scrutin étudier dans ce document, seul la méthode Borda et le 
  jugement majoritaire échape à ce théorème, car ils n'utilisent pas de classement des 
  options par les électeurs.

  \section{Quasi unanimité du vote}

  La quasi unanimité du vote, c'est que lorsque que presque l'ensemble des électeurs préfèrent 
  un option (A) à une autre option (B), alors cette deuxième option (B) n'est jamais élu.
  Il y a différente méthode de vote qui ne respecte pas la quasi unanimité du vote.

  \section{Manipulabilité et vote utile}

  \subsection{Vote utile}%
  \label{diff:comp:util:util}

  Le vote utile est le fait de changer son ordre de préférence entre ses candidats favoris pour
  permettre à l'un de ses candidats favoris de remporté l'élection. 
  Je choisi de ne pas considérer comme vote utile le fait de voter pour un candidat que l'on ne
  souhaite pas voir élu, afin que son candidat favori soit élu, j'ai classée cette pratique 
  dans la manipulabilité du résultat (voir \aref{diff:comp:util:manip}).

  Dans le cadre de cette définition le vote utile n'est en soit pas mauvais.
  Il permet d'aider à la recherche d'un consensus en impliquant les citoyen dans la réflexion
  et donc sans se basé exclusivement sur des maths que toutes la populations n'est pas forcement
  en mesure de comprendre.

  %\begin{nota}[élection présidentiel française 2022]{Exemple de vote utile}
  %  Par exemple lors de l'élection présidentiel française de 2022, certaines personnes de gauche
  %  radical ont choisi à voter pour J.-L. Mélanchon au premier tour alors même qu'ils aurait 
  %  plutôt voulu voter pour des candidats plus à gauche (P. Poutou).
  %  Ce n'est pas forcement une mauvais chose, car leur objectif en changeant leur vote n'est pas
  %  de que P. Poutou gagne, mais bel et bien que J.-L. Mélanchon atteigne le second tour.
  %  \todo{compléter example manipulation}
  %\end{nota}

  \todo{Faire exemple avec graphique}

  \subsection{Manipulabilité du résultat et corruption}%
  \label{diff:comp:util:manip}%
  \label{diff:comp:util:corruption}

  La manipulation du vote est le fait de ne pas voter pour ses candidats/options favoris, mais 
  pour un candidat que l'on ne veux pas voir au pouvoir dans l'objectif de faire en sorte que 
  notre candidat favori soit finalement élu.

  \todo{Faire exemple avec graphique}

  Le corruption du scrutin corresponds à l'achat\footnote{Par achat on prends en compte 
  tout échange visant à limiter le choix libre du vote allant d'un petit avantage à la menace 
  de mort sur le/la votant·e ou ses proches.} auprés des électeurs de leur vote.
  La corruption est légerement différent de la manipulation, dans le sens où le candidat
  que l'on cherche à élire à la fin n'est pas forcement celui qui est préférer par le votant.
  Tout les systèmes de vote est vulnérable à ce genre de manipulation, mais un bon système de 
  vote devrait necessité une très forte corruption pour que le résultats final soit impacté.
  Certaine technique peuvent être mise en place pour limiter la corruption des votants (secret
  du vote, impossibilité de prouver pour qui on a voter, liberté de la presse, …), mais 
  ces techniques ne peuvent pas garantir l'absence de correciont.
  
  %\begin{nota}[élection présidentiel française 2022]{Exemple de tentative de manipulation du vote}
  %  Lors du premier tours 
  %  \todo{compléter example}
  %\end{nota}

  \todo{Faire exemple avec graphique}

  \section{Autre critère non pris en compte}
  \subsection{Votes blancs, nuls et abstentions}

  Les votes blancs et nuls devrait toujours être compté, ne serais-ce que pour des raisons de
  statistique.
  Savoir qu'il y a eu 90\% de vote blanc permet au politique de voir qu'il y a un problème, même
  s'il n'y a aucun moyen de savoir ce qui c'est passé, au simple regard de cette statistique.

  Dans le cas où l'on veux prendre en compte les votes blancs, nuls ou abstentions, il est primordial
  de faire attention que cela ne privilégie pas une option par rapport à une autre.

  \begin{nota}{Décompte de vote blancs non-neutre}
    Lors d'un référendum contre une désicion du gouvernement, il faut qu'il y ai plus de 50\% 
    de vote contre pour que la désicion soit révoquée.

    Dans le cas où les votes blancs, nuls ou bien les abstansions sont pris en compte dans 
    le nombre de personne ayant validement voter, cela augmente le nombre de vote contre 
    qu'il y a besoin pour refusé le choix du gouvernement.

    De ce fait, ces «votes» vont donc dans le sens du gouvernement et ne sont donc pas neutre. 

    Si les votes blancs/nuls font juste baissé le quorum, et que si le quorum n'est pas atteint
    une seconde votation a lieu, alors, c'est un usage plus correct de ces voix.

  \end{nota}


  \subsection{Corps électoral}

  Pour avoir une bonne représentation de ce que veux la population, avoir un faire taux de votes
  blancs, nuls et d'abstansions n'est pas pas suffisant.
  En effet dans la majorité des pays, une grande partie de la population n'a pas le droit de vote.

  \begin{nota}{Droit de vote en Suisse}
    En 2015, 37\%\footnote{au niveau fédéral, source: \url{https://www.swissinfo.ch/fre/democratiedirecte/élections-2019\_qui-peut-voter-en-suisse--et-qui-ne-peut-pas-/45264666}}
    de la population Suisse n'avait simplement pas le droit de vote.
    \begin{enumerate}
      \item Les étrangers comme dans beaucoup de pays.
      \item Les jeunes jusqu'à 18 ans, cette age peux varier en fonction du pays
      \item Dans certain pays (mais pas la Suisse), les personnes incarcérée, n'ont pas 
        le droit de vote.
      \item Et enfin toutes personne sous curatelle de portée générale ne peuvent 
        également pas voter.
      \item Et enfin toutes personne sous curatelle de portée générale ne peuvent 
      \item Jusqu'à récemment certain pays\footnote{L'Arabie Saoudite a donné le droit de vore au femme en 2011, source: \url{https://fr.wikipedia.org/wiki/Droit\_de\_vote\_des\_femmes}} ne donnait pas le droit de vote aux femmes.
    \end{enumerate}
  \end{nota}

  Ces diverses restrictions fait qu'une partie de la population vis dans un pays, mais n'a
  pas la possibilité de participé au décision collective.
  Lorsqu'on essaye d'interpréter un scrutin réel, il faut donc prendre en compte cela.
  Un candidat élu à 70\% n'est pas élus à 70\% de la population, ni même à 70\% des personnes
  ayant le droit de vote, ni même à 70\% des personnes inscrites sur les listes électorales,
  même pas à 70\% des votants, mais à 70\% des votants n'ayant pas voter blanc et dont leur 
  bulletin n'a pas été considérer comme nul.
  En fonction du pays et de la votation cela peut avoir une grande importance pour savoir ce que
  veux la population général.

  Ne pas oublier que certain pays donne le droit vote aux personnes ayant la nationalité mais 
  résidant à l'étranger.
  Le résultat ne corresponds donc pas à ce que veulent les résidants, mais est influancer par ce
  que veulent des personnes vivants à l'étranger.

  \begin{table}[h]
    \caption{Suisse: Initiative populaire «~99~\%~»}%
    \label{tab:votation:99pourcent}%
    % voir : https://www.bfs.admin.ch/bfs/fr/home/statistiques/politique/votations/annee-2021/2021-09-26/initiative-99-pour-cent.html
    % voir : https://www.bfs.admin.ch/bfs/fr/home/statistiques/population/effectif-evolution/population.html
    \begin{center}
      \begin{tabular}{lrccc}
        \hline
        & & Électeur & Pour [\%] & Contre [\%] \\
        \hline
        \hline
        Résultat officiel        &              & 2~810~307 & 35,12          & 64,88          \\
        Avec vote blanc          & $+58~499$    & 2~868~806 & 34,41 -- 36,45 & 63,55 -- 65,59 \\
        Avec vote nuls           & $+14~073$    & 2~882~879 & 34,24 -- 36,76 & 63,24 -- 65,76 \\
        Sans abstentions         & $+2~636~289$ & 5~519~198 & 17,88 -- 66,96 & 33,04 -- 82,12 \\
        Populatation tot. (2020) & $+3~151~132$ & 8~670~300 & 11,38 -- 78,97 & 21,03 -- 88,62 \\
        \hline
      \end{tabular}
    \end{center}
  \end{table}

  La Suisse a voter le 26 septembre 2021 sur l'initiative «~99~\%~».
  Cette initiative a été largement refusée à plus de 64\%.
  Le \aref{tab:votation:99pourcent} montre les pourcentages de la populations possible dans le 
  cas où personne n'aurait voter blanc, auncun bulletin nul décompter, sans abstention et 
  finalement si l'ensemble de la population avait le droit de vote et votait.
  Par exemple on sais avec certitude que 21,03\% de la population était contre 
  (l'initiative a été rejetée), mais la propotion de personne contre peux varier entre 
  21,03\% et 88,62\% si l'on ne regarde que le résultat du scrutin.
  La population prise en compte et celle de 2020 (une année avant le scrutin).

  \subsection{Reflexion et débat}

  Certain scrutin se déroule en plusieurs tours, il peut être interessant de demander à la 
  population de classer tous les candidats/options et de simuler tous les tours.
  Cette technique permet aux électeurs de ne se déplacer qu'une seul fois aux urnes.
  Malheureusement, en simulant les tours, on supprime également la possibilité d'avoir un
  débat et donc que les électeurs puisse changer d'avis.
  Les compromis entre candidat sont également impossible.

  Un scrutin à plusieurs tours réel permet d'améliorer le débat politique, même si un scrutin 
  à 10 lasserait certainement plus que le bénéfisse possible.

  \chapter{Les scrutins}

  \begin{important}[$50\% + 1$ versus $ > 50\%$]{Définition de la majorité}
    On trouve fréquement l'expression $50\%+1$ pour désigner le nombre de vote requis pour
    obtenir la majorité, alors qu'il faux juste plus de $50\%$ des voix pour la majorité.
    Même si cette distinction parrait minime, pour des scrutins avec un nombre d'électeur 
    impair c'est un point important.

    Si A a 4 voix, B a 1 voix et C a 2 voix, on a un total de 7 électeurs.
    Dans un cas pour avoir la majorité nous devons obtenir le nombre de voix suivant
    \[ \text{plus que }50\%\text{ de }7 = \frac{7}{2} = 3.5 \]
    alors que dans l'autre cas
    \[ \text{au moins }(50\%\text{ de }7) + 1 = \frac{7}{2} + 1 = 4.5 \]

    Nous avons donc un cas où il faut un score $ > 3.5 $ et un autre cas où il faut un score 
    $ \geqslant 4.5 $.
    Le candidat A se retrouve donc élu avec un définition et non-élu avec l'autre.

    Dans ce document seul la définition $> 50\%$ a été utilisé
  \end{important}

  \section{Scrutin proportionnel plurinominal}

  Les électeurs votent pour une liste (de candidats) de leur choix.
  Chaque liste remporte un nombre de siège proportionnel au nombre de voix reçue.
  Les sièges sont finalement distribué dans chaque liste soit par ordre de présence dans
  la liste, ou bien en fonction du nombre de voix de chaque candidat dans la liste.

  \paragraph{Scrutin avec seuil}

  Lors de certaine élection il peut y avoir un seuil à atteindre pour qu'un liste puisse 
  prétendre à un siège.
  C'est-à-dire qu'il faut avoir par exemple les voix necessaire pour 3 sièges pour permettre
  à la liste de rentré dans le groupe de liste recevant des sièges.
  S'il n'y a pas de seuil, chaque liste avec un nombre de voix suffisant pour 1 siège participe 
  à la répartition.
  
  \paragraph{Préférentiel ou listes bloquées}

  En listes bloquées, les sièges sont répartis au sein des listes par ordre d'apparision 
  des candidats.
  En préférentiel, les sièges sont répartis en fonction des préférences des électeurs.
  Ces derniers peuvent changer l'ordre des candidats, biffant/rajoutant certain ou même
  en cumulant certain candidat de plusieurs foix.
  
  \paragraph{Répartition des sièges}

  Les candidats ne pouvant pas être élu à moitié, diverse méthode existe pour répartir
  les sièges entre les diverses listes.
  Chaques méthodes peuvent donner des resultats différents.
  
  \section{Scrutin uninominal majoritaire}\label{diff:scrutin-majoritaire}
  
  Chaque votant vote pour son option préférée.
  Pour chaque option on compte le nombre d'électeur l'ayant choisi, l'option avec le plus 
  d'électeur l'emporte.
  
  \subsection{Majorité relative ou absolue}
  \subsubsection{Majorité absolue}
  Dans le cas d'un scrutin à majorité absolue, l'option avec plus de 50\% de vote est déclarée
  vainqueure.
  Dans le cas où aucune options n'atteint les 50\%, une tour supplémentaire est généralement
  organiser.
  Ce tour supplémentaire peux être à la majorité relative dans une volonté de limité le nombre
  de tour.
  
  \begin{nota}{Pourquoi 50\%~?}
    Pour certain scrutin plus important, le seuils est plus élevé que 50\%, exemple 60\% ou 70\%. 
    En quoi est-ce que 50\% est un bon choix, est-ce que ça veux dire que le choix de la moitié
    des votants n'as pas importance~? 
  \end{nota}
  
  \subsubsection{Majorité relative}
  Dans le cas d'un scrution à majorité relative, l'option avec le plus de voix est choisie, 
  quelque soit son pourcentage de voix. Il n'y a donc pas besoin d'un second tours s'il n'y 
  a pas d'ex-aequo. Ce scrutin est décris dans l'algorithme \aref{scrutin:maj-uni-1t}.

  \begin{algorithm}
    \caption{Scrutin majoritaire uninominal à 1 tour}%
    \label{scrutin:maj-uni-1t}
    \begin{algorithmic}[1]
      \REQUIRE{$votes$[n° du votant] = n° du choix favori}
      \ENSURE{Liste des n° des choix vainqueurs}
      \FORALL{$vote \leftarrow votes$}
      \STATE{$total[vote]$++}
      \ENDFOR{}
      \RETURN{Les indexes de max$(total)$}
    \end{algorithmic}
  \end{algorithm}


  \subsection{Nombre de tour}

  Lors d'un scrutin à la majorité relative, un seul tour est necessaire.
  Pour la variante à la majorité absolue, un candidat n'est pas forcement élu lors du
  premier tour. Dans ce dernier cas un tour suplémentaire doit être organisé, les options
  les plus connus sont~: 
  \begin{itemize}
    \item Garder les 2 meilleures candidats. C'est le dernier tour, il y en a forcement un 
      qui aura plus de 50\%\footnote{S'il n'y a pas ex-aequo et que les votes blancs/nuls 
      et abstentions n'influence pas le résultat}. C'est ce qui est utilisé pour l'élection
      présidentiel française. Ce scrutin est décris dans l'algorithme \aref{scrutin:maj-uni-2t}.
    \item Éliminer le pire des candidats, s'il y a encore plus que 2 candidats, il peut avoir
      encore d'autre tour. C'est ce qui est utilisé pour l'élection de chaque membre du
      conseil fédéral à partir du 3\up{e} tour.
  \end{itemize}

  Entre chaque tours les candidats ont habituellement la posibilité de se retirer du scrutin.
  
  \begin{nota}{Qui éliminer~?}
    Habituellement, les candidats/options avec le moins de voix sont éliminé, mais
    est-ce vraiment leurs électeurs qui sont le plus suceptible de changer d'avis~?
    Est-ce qu'un électeur d'un des 2 candidat proche mutuellement n'aurait pas plus
    de facilité à changer son vote~?
  \end{nota}

  \begin{algorithm}
    \caption{Scrutin majoritaire uninominal à 2 tour}%
    \label{scrutin:maj-uni-2t}
    \begin{algorithmic}[1]
      \REQUIRE{$votes$[n° du votant] = Liste des choix par ordre de préférence décroisante}
      \ENSURE{Liste des n° des choix vainqueurs}
      \STATE{$majorité \leftarrow \frac{\text{len}(votes)}{2} $}
      \FORALL{$vote \leftarrow votes$}
      \STATE{$choix \leftarrow vote[0]$ \COMMENT{Premier choix du votant}}
      \STATE{$total[choix]$++}
      \ENDFOR{}
      \IF{$\text{max}(total) > majorité$}
      \RETURN{Index de max$(total)$}
      \ENDIF{}
      \STATE{}
      \STATE{$restant \leftarrow $ index des 2 valeurs maximal de $total$}
      \STATE{vider $total$}
      \FORALL{$vote \leftarrow votes$}
      \STATE{$choix \leftarrow $ première valeur de $vote$ se trouvant dans $restant$}
      \STATE{$total[choix]++$}
      \ENDFOR{}
      \IF{$\text{max}(total) > majorité$}
      \RETURN{Index de max$(total)$}
      \ELSE{}
      \STATE{\COMMENT{Les 2 choix restants sont ex-aequo}}
      \RETURN{$restant$}
      \ENDIF{}
    \end{algorithmic}
  \end{algorithm}

  \subsection{Élection étrange}

  \begin{table}[h]
    \begin{center}
      \caption{Cas limites d'un scrutin à la majorité relative (à 1 tour)}%
      \label{fig:diff:maj1:caslim1}
      \adjustbox{valign=t}{%
        \begin{subtable}[h]{0.45\textwidth}
          \centering
          \caption{Préférences des électeurs}%
          \label{fig:diff:maj1:caslim1:pref}
          \inputMake{_dyn/scrutin/maj1-caslim-1.votes.tex}
        \end{subtable}
      }
      \adjustbox{valign=t}{%
        \begin{subtable}[h]{0.45\textwidth}
          \centering
          \caption{Résultats du scrutin}%
          \label{fig:diff:maj1:caslim1:result}
          \inputMake{_dyn/scrutin/maj1-caslim-1.maj1.tex}
        \end{subtable}
      }
    \end{center}
  \end{table}

  Les tableaux \aref{fig:diff:maj1:caslim1} montre les préférences d'un corps 
  électoral et le résultat d'une votation avec un scrutin majoritaire à 1 tour.

  On peux constater plus de 60\% des électeurs (ii et iii) préférent B à A ou C à A. 
  Cependant comme A a fait plus de voix que B ou C, c'est donc cette option qui est élue.
  Si B ou C s'était retirer avant le scrutin ou s'il y avait eu une candidature commune, A aurait été largement battue. 

  \subsection{Évaluation}
  \subsubsection{À 2 tours}
  \tabcritere%
    {\cellcolor{red}trés mauvais}{Une grande part de la population (>50\%) peut être contre l'élu}%
    {\cellcolor{red}mauvais}{Les petits candidats «~vols~» des voix aux autres}%
    {\cellcolor{red}mauvais}{Si les voix sont sufisament divisé dans un des camps}%
    {\cellcolor{red}mauvais}{En empéchant un candidat d'accéder au second tour}

  \subsubsection{À 1 tour}
  \tabcritere%
    {\cellcolor{red}trés mauvais}{Une grande part de la population (>50\%) peut être contre l'élu}%
    {\cellcolor{red}mauvais}{Les petits candidats «~vols~» des voix aux autres}%
    {\cellcolor{red}mauvais}{Si les voix sont sufisament divisé dans un des camps}%
    {\cellcolor{orange}moyen}{Si les candidats sont relativement prôche dans leurs scores}

  \section{Méthode Borda}
  
  Chaques votants classent tous ou parties des options dans l'ordre de leur préférence.
  Les candidats reçoivents des points en fonction de leur position dans l'ordre de chaque bulletin.
  Le/la vainqueur·e est celui qui as obtenu le plus de points.

  \subsection{Distribution des points}

  Dans la description de cette méthode de vote faite par J.-C. Borda donnait 1 point au
  candidat classé en dernier, puis un point supplémentaire pour l'avant-dernier et ainsi
  de suite jusqu'au premier candidat.
  Dans la pratique actuelle~\cite{emerson_original_2013} les points sont distribué du 
  premier au dernier $(n, n-1, …, 1)$.
  Si tous les électeurs classe l'ensemble des candidats, alors ces 2 méthode donne le même 
  résultat. Cependant lorsqu'on autorise le classement partiel, il y a des différence.

  Pour évité toute ambigüiter, je désignerais la méthode initialement décrite par borda, 
  la méthode Borda avec distribution des points \textbf{classiques} et la méthode distribuant
  les points à partir du premier candidat, distribution des points \textbf{morderne}.

  \begin{table}
    \begin{center}
      \caption{Méthode Borda différence entre les méthodes de distribution des points}%
      \label{fig:diff:borda:caslim2}
      \adjustbox{valign=t}{
        \begin{subtable}[h]{0.35\textwidth}
          \centering
          \caption{Préférences des électeurs}
          \inputMake{_dyn/scrutin/borda-caslim-2.votes-c.tex}
        \end{subtable}
      }
      \adjustbox{valign=t}{
        \begin{subtable}[h]{0.25\textwidth}
          \centering
          \caption{Points classique}
          \inputMake{_dyn/scrutin/borda-caslim-2.bordaclasse.tex}
        \end{subtable}
      }
      \adjustbox{valign=t}{
        \begin{subtable}[h]{0.25\textwidth}
          \centering
          \caption{Points moderne}
          \inputMake{_dyn/scrutin/borda-caslim-2.bordatot.tex}
        \end{subtable}
      }
    \end{center}
  \end{table}

  Comme le montre les tableaux \aref{fig:diff:borda:caslim2}, la méthode classique pousse 
  les votants à classer un maximum de candidat et donc aide à la recherche d'un 
  compromis convenant à tout le monde.
  Alors que la méthode moderne pousse plus à ne classer que le candidat favori, si on grande
  partie des votants choisissent de faire cela, on se retrouve dans une situation proche 
  d'un scrutin majoritaire à majorité relative.

  Les deux méthodes de distribution des points sont décrits dans les algorithmes \aref{scrutin:borda-classique} et \aref{scrutin:borda-moderne}.

  \begin{algorithm}
    \caption{Méthode Borda (avec distribution des points classique)}%
    \label{scrutin:borda-classique}
    \begin{algorithmic}[1]
      \REQUIRE{$votes$[n° du votant] = Liste des choix par ordre de préférence décroisante}
      \ENSURE{Liste des n° des choix vainqueurs}
      \FORALL{$vote \leftarrow votes$}
      \STATE{$point \leftarrow \text{len}(vote)$}
      \FORALL{$option \leftarrow vote$}
      \STATE{$total[option] \leftarrow total[option] + point$}
      \STATE{$point--$}
      \ENDFOR{}
      \ENDFOR{}
      \RETURN{Les indexes de max$(total)$}
  \end{algorithmic}
  \end{algorithm}
  
  \begin{algorithm}
    \caption{Méthode Borda (avec distribution des points moderne)}%
    \label{scrutin:borda-moderne}
    \begin{algorithmic}[1]
      \REQUIRE{$votes$[n° du votant] = Liste des choix par ordre de préférence décroisante}
      \ENSURE{Liste des n° des choix vainqueurs}
      \FORALL{$vote \leftarrow votes$}
      \STATE{$point \leftarrow $ Nombre d'option disponible}
      \FORALL{$option \leftarrow vote$}
      \STATE{$total[option] \leftarrow total[option] + point$}
      \STATE{$point--$}
      \ENDFOR{}
      \ENDFOR{}
      \RETURN{Les indexes de max$(total)$}
    \end{algorithmic}
  \end{algorithm}

  \subsection{Manipulabilité de l'élection}
  \begin{table}[h]
    \begin{center}
      \caption{Cas limites d'un scrutin utilisant la méthode Borda}%
      \label{fig:diff:borda:caslim1}
      \adjustbox{valign=t}{
        \begin{subtable}[h]{0.40\textwidth}
          \centering
          \caption{Préférences des électeurs}%
          \label{fig:diff:borda:caslim1:A}
          \inputMake{_dyn/scrutin/borda-caslim-1A.votes-c.tex}
        \end{subtable}
      }
      \adjustbox{valign=t}{
        \begin{subtable}[h]{0.20\textwidth}
          \centering
          \caption{Points classique}%
          \label{fig:diff:borda:caslim1:B}
          \inputMake{_dyn/scrutin/borda-caslim-1A.bordaclasse.tex}
        \end{subtable}
      }
      \adjustbox{valign=t}{
        \begin{subtable}[h]{0.20\textwidth}
          \centering
          \caption{Points moderne}%
          \label{fig:diff:borda:caslim1:C}
          \inputMake{_dyn/scrutin/borda-caslim-1A.bordatot.tex}
        \end{subtable}
      }\\[1em]
      \adjustbox{valign=t}{
        \begin{subtable}[h]{0.40\textwidth}
          \centering
          \caption{Préférences des électeurs}%
          \label{fig:diff:borda:caslim1:D}
          \inputMake{_dyn/scrutin/borda-caslim-1B.votes-c.tex}
        \end{subtable}
      }
      \adjustbox{valign=t}{
        \begin{subtable}[h]{0.20\textwidth}
          \centering
          \caption{Points classique}%
          \label{fig:diff:borda:caslim1:E}
          \inputMake{_dyn/scrutin/borda-caslim-1B.bordaclasse.tex}
        \end{subtable}
      }
      \adjustbox{valign=t}{
        \begin{subtable}[h]{0.20\textwidth}
          \centering
          \caption{Points moderne}%
          \label{fig:diff:borda:caslim1:F}
          \inputMake{_dyn/scrutin/borda-caslim-1B.bordatot.tex}
        \end{subtable}
      }
    \end{center}
  \end{table}

  Les tableaux~\ref{fig:diff:borda:caslim1:A} à~\aref{fig:diff:borda:caslim1:C} indique les 
  préférences des électeurs, les tableaux~\ref{fig:diff:borda:caslim1:D} 
  à~\ref{fig:diff:borda:caslim1:F} montrent la même élection avec une tentative de manipulation 
  du vote.

  Nous pouvons y constater que lorsqu'on utilise le décompte de points moderne, une petite 
  proportion des électeurs (16\%) arrivent à faire basculer le resultat à leur avantage.


  \subsection{Évaluation}
  \subsubsection{Borda avec distribution des points classique}
  \tabcritere%
    {\cellcolor{green}bien}{L'avis des personnes n'ayant pas voter pour le gagnant influance le résultat.}%
    {\cellcolor{orange}moyen}{Les petits candidats offrent un avantage aux candidat proche. Dans le cas d'un élection équilibrée, ce n'est pas un problème.}%
    {\cellcolor{orange}moyen}{Ne respecte pas ce critère, mais essaye de minimiser les personnes très mécontante du résultats.}%
    {\cellcolor{green}bien}{Les électeurs on tout intérêt à donner leur vrai préférences en général}
  \subsubsection{Borda avec distribution des points moderne}
  \tabcritere%
    {\cellcolor{green}bien}{L'avis des personnes n'ayant pas voter pour le gagnant influance le résultat.}%
    {\cellcolor{green}bien}{Les candidats non-classés en premier n'influance pas les points donnés au premier.}%
    {\cellcolor{orange}moyen}{Ne respecte pas ce critère, mais essaye de minimiser les personnes très mécontante du résultats.}%
    {\cellcolor{red}mauvais}{Les électeurs on intérêt à classé moins bien leurs second choix pour avantage leur candidat favori.}


  \section{Méthode de Condorcet}

  Les électeurs classent les candidats par ordre de préference. 
  Puis est simuler des duels entre chaques candidats au scrutin majoritaire.
  Le candidat ayant battu l'ensemble des autres candidats est élu.

  \subsection{Paradoxe de Condorcet}

  Le paradoxe de Condorcet, lors d'un vote par classement de 3 options (A, B, C).
  C'est qu'il peut avoir une majorité de votant préfèrant C à A, une autre majorité 
  préférant B à C, et qu'une dernière majorité préfèrent A à B.
  Dans un tel cas, quelque soit l'option retenue, il y a toujours plus de 50\% des votants
  qui serait pour changer d'option.

  Dans le cas où il y a un paradoxe de Condorcet, d'autres méthodes doivent être utilisées pour désigner
  le gagnant.
  Cela peut être un autre méthode de vote ou un algorithme spécifique.
  \todo{Retrouver l'exemple réel du paradoxe lors des élections présidentiel française}
  \subsection{Élection étrange}
  \begin{table}[h]
    \begin{center}
      \caption{Scrutin à la Condorcet ne prenant pas en compte les perdants}%
      \label{fig:diff:condorcet:caslim1}
      \adjustbox{valign=t}{%
        \begin{subtable}[h]{0.45\textwidth}
          \centering
          \caption{Préférences des électeurs}%
          \label{fig:diff:condorcet:caslim1:pref}
          \inputMake{_dyn/scrutin/condorcet-caslim-1.votes-c.tex}
        \end{subtable}
      }
      \adjustbox{valign=t}{%
        \begin{subtable}[h]{0.45\textwidth}
          \centering
          \caption{Résultats du scrutin}%
          \label{fig:diff:condorcet:caslim1:result}
          \inputMake{_dyn/scrutin/condorcet-caslim-1.condorcet.tex}
        \end{subtable}
      }
    \end{center}
  \end{table}

  Les tableaux~\aref{fig:diff:condorcet:caslim1} montrent un scrutin effectuer avec la méthode de Condorcet,
  donnant vainqueurs A.
  L'option C a été classée en première ou deuxième position par l'ensemble des électeurs, alors que l'option
  vainqueure (A) à été réfusé sechement par un tiers des électeurs.
  Dans ce scrutin l'avis d'un tiers de la population a purement été ignoré.

  \subsection{Évaluation}
  \tabcritere%
    {\cellcolor{red}mauvais}{Seul l'avis de la majorité est prise en compte, l'avis des 49\% restants est ignoré}%
    {\cellcolor{green}bien}{Lorsqu'il y a un unique vainqueur, la méthode est mathématiquement résistant au petit candidat}%
    {\cellcolor{green}bien}{Lorsqu'il y a un unique vainqueur, ce critère est forcement respecté}%
    {\cellcolor{green}bien}{Le système de vote pousse les électeurs à donner leur vrai préférence}

  \todo{retrouver papier à propos de la résistance au petit candidat}
  \todo{Évaluation méthode de Condorcet randomisé}

  \section{Vote par approbation}

  Chaque électeur indique le ou les candidats qu'il trouve accéptable d'être élu.
  Le candidat ayant reçu le plus d'approbation est élu.
  Ce système est facile à mettre en place pour les élections actuelles, car il suffirait d'autoriser
  les électeurs à glisser plusieurs bulletin différent dans l'urne.

  \subsection{Élection étrange}

  \begin{table}[h]
    \begin{center}
      \caption{Vote par approbation: cas étrange}%
      \label{fig:diff:appro:caslim1}
      \adjustbox{valign=t}{
        \begin{subtable}[h]{0.45\textwidth}
          \centering
          \caption{Préférences des électeurs}
          \inputMake{_dyn/scrutin/apro-caslim-1.votes.tex}
        \end{subtable}
      }
      \adjustbox{valign=t}{
        \begin{subtable}[h]{0.45\textwidth}
          \centering
          \caption{Résultats}
          \inputMake{_dyn/scrutin/apro-caslim-1.apro.tex}
        \end{subtable}
      }
    \end{center}
  \end{table}

  Dans l'élection représentater dans les tableaux \aref{fig:diff:appro:caslim1}, l'option
  C gagne alors que 75\% de l'électorat lui préfère largement A et 75\% lui préfère B.

  \subsection{Évaluation}
  \tabcritere%
    {\cellcolor{green!25!yellow}correct}{L'avis des perdants est pris en compte, cependant on ne peux indiquer que l'on ne veux pas de 2 candidats, mais que malgrès tous, l'un des deux nous est moins défavorable.}%
    {\cellcolor{green}bien}{Chaque candidat as un score propre, ne dépendant pas des autres candidats.}%
    {\cellcolor{red}mauvais}{Ne respecte pas ce critère}%
    {\cellcolor{green}bien}{Cette méthode est faiblement manipulable.}

  \section{Jugement majoritaire}

  Chaque électeur donne une évaluation à chaque candidat.
  Chaque candidat reçoit un évaluation final qui corresponds à l'évaluation médianne qu'il a reçu (donc il y a au moins 
  50\% des électeurs qui on donner cette évaluation ou plus).
  Le candidat élus est celui avec l'évaluation la plus élevée.
  Ce scrutin est décris dans l'algorithme \aref{scrutin:jugement-maj}.

  L'évaluation peut être numérique, par exemple un note sur 20. Mais il n'est absolument pas sur qu'une note de 12/20 ai
  la même signification pour tous les électeurs, ce qui pose problème pour intérpréter les résultats. 
  Une autre technique consiste à donner des appréciations à chaque candidat (excellent, bien, passable, …), l'avantage
  est que si les termes en français sont bien choisi, ils ont à peut près la même signification pour tous.

  Pour les exemples de ce document, ce sont des appréciations qui on été choisie~: 

  \begin{center}
  \colorbox{green}{Excellent} $\succ$
  \colorbox{green!50!yellow}{Bien} $\succ$
  \colorbox{green!25!yellow}{Correct} $\succ$
  \colorbox{yellow}{Passable} $\succ$
  \colorbox{orange}{Insufisant} $\succ$
  \colorbox{red}{A rejeter}
  \end{center}
  
  \begin{algorithm}[H]
  \caption{Scrutin au jugement majoritaire}%
  \label{scrutin:jugement-maj}
  \begin{algorithmic}[1]
    \REQUIRE{$votes$[n° du votant][n° de l'option] = Le jugement de l'option (5 = parfait; 0 = À rejeter)}
    \ENSURE{Liste des n° des choix vainqueurs}
    \STATE{$majorité \leftarrow \frac{\text{len}(votes)}{2}$}
  \FORALL{$vote \leftarrow votes$}
    \FORALL{$option \leftarrow vote$}
      \STATE{$jugement \leftarrow vote[option]$}
      \STATE{$total[option][jugement] ++$}
    \ENDFOR{}
    \ENDFOR{}
  % TODO
  \end{algorithmic}
  \end{algorithm}
  \todo{Écrire la fin de l'algorithme}

  \begin{table}[H]
    \begin{center}
      \caption{Jugement majoritaire: manipulabilité du vote}%
      \label{fig:diff:jugmaj:caslim1}
      \adjustbox{valign=t}{
        \begin{subtable}[h]{0.45\textwidth}
          \centering
          \caption{Préférences des électeurs}
          \inputMake{_dyn/scrutin/jugmaj-caslim-1A.votes-n.tex}
        \end{subtable}
      }
      \adjustbox{valign=t}{
        \begin{subtable}[h]{0.45\textwidth}
          \centering
          \caption{Résultats}
          \inputMake{_dyn/scrutin/jugmaj-caslim-1A.jugmaj.tex}
        \end{subtable}
      }\\[1em]
      \adjustbox{valign=t}{
        \begin{subtable}[h]{0.45\textwidth}
          \centering
          \caption{Tentative de manipulation}
          \inputMake{_dyn/scrutin/jugmaj-caslim-1B.votes-n.tex}
        \end{subtable}
      }
      \adjustbox{valign=t}{
        \begin{subtable}[h]{0.45\textwidth}
          \centering
          \caption{Résultats}
          \inputMake{_dyn/scrutin/jugmaj-caslim-1B.jugmaj.tex}
        \end{subtable}
      }
    \end{center}
  \end{table}
  % TODO compresser ce graphique
  \subsection{Vote utile}

  Les tableaux \aref{fig:diff:jugmaj:caslim1} montrent un élection normal, puis la même élection avec une tentative
  de manipulation de la part de 20\% des électeurs. 
  Ces électeurs ont choisi de modifier leur vote en donnant une mauvaise jugement au candidat B qu'ils trouvaient 
  initialement correct.

  \subsection{Évaluation}
  \tabcritere%
    {\cellcolor{red}mauvais}{L'avis des 49\% de la population est ignorée.}%
    {\cellcolor{green}bien}{Chaque candidat a une évaluation propre ne dépendant pas des autres candidats.}%
    {\cellcolor{red}mauvais}{Seul l'avis médian est pris en compte}%
    {\cellcolor{green!25!yellow}correct}{La manipulation n'est pas évidante, car il faut modifier la médian}

  \section{Méthode de Coombs et vote alternatif}

  Chaques votants classent tous ou parties des options dans l'ordre de leur préférence.
  Puis on simule un scrutin à la majorité absolue sans limite de nombre de tours, en retirant le pire candidats
  entre chaque tour.

  \subsection{Vote alternatif}
  Aussi appelé vote préférencielle transférable ou Méthode de R, le vote alternatif retire à chaque 
  tour le candidat ayant été classé le moins souvant en première position.
  C'est donc un simulation exacte d'un scrutin à la majorité absolue sans limite de nombre de tours, 
  mais en retirant la posibilité de réflexion entre chaque tour. 
  Il a donc les même désavatage que ce dernier (voir \aaref{diff:scrutin-majoritaire}).

  \subsection{Méthode de Coombs}
  La méthode de Coombs retire à chaque tour le candidat ayant été classé le plus souvant en dernière position.
  Cette méthode à l'avantage 

  \subsection{Exemple}

  \begin{table}[h]
    \begin{center}
      \caption{Example de scrutin méthode de Coombs et vote alternatif}%
      \label{fig:diff:coombs:caslim1}
      \adjustbox{valign=t}{%
        \begin{subtable}[h]{0.45\textwidth}
          \centering
          \caption{Préférences des électeurs}%
          \label{fig:diff:coombs:caslim1:pref}
          \inputMake{_dyn/scrutin/maj1-caslim-1.votes.tex}
        \end{subtable}
        }\\[1em]
        \adjustbox{valign=t}{%
          \begin{subtable}[h]{0.45\textwidth}
            \centering
            \caption{Coombs}%
            \label{fig:diff:coombs:caslim1:coombs}
            \inputMake{_dyn/scrutin/maj1-caslim-1.coombs.tex}
          \end{subtable}
        }
        \adjustbox{valign=t}{%
          \begin{subtable}[h]{0.45\textwidth}
            \centering
            \caption{Vote alternatif}%
            \label{fig:diff:coombs:caslim1:alternatif}
            \inputMake{_dyn/scrutin/maj1-caslim-1.alternatif.tex}
          \end{subtable}
        }
    \end{center}
  \end{table}

  Les tableaux \aref{fig:diff:coombs:caslim1} montre qu'il y a une réel différence entre les 2 méthodes

  \subsection{Évaluation}
  \subsubsection{Méthode de Coombs}
  \tabcritere%
    {\cellcolor{green!25!yellow}correct}{En retirant le candidat le plus souvant en dernière position, on prends en compte l'avis des perdants. Cela ne suffit pas pour s'assurer que qu'il n'y a pas de meilleures candidats dans le cas où un candidat est élu à 51\%}%
    {\cellcolor{red}mauvais}{Les petits candidats «~vols~» des voix aux autres}%
    {\cellcolor{red}mauvais}{Si les voix sont sufisament divisé dans un des camps}%
    {À évaluer}{}
  \subsubsection{Vote alternatif}
  \tabcritere%
    {\cellcolor{red}mauvais}{L'avis de ceux n'ayant pas choisi le vainqueur n'est pas pris en compte.}%
    {\cellcolor{red}mauvais}{Les petits candidats «~vols~» des voix aux autres}%
    {\cellcolor{red}mauvais}{Si les voix sont sufisament divisé dans un des camps}%
    {À évaluer}{}


  
  %\begin{nota}[by me]{Ma note}\lipsum[1][1-5]\end{nota}
  %\begin{question}[by me]{Ma question}\lipsum[2][1-5]\end{question}
  %\begin{important}[by me]{Ma not imp!}\lipsum[3][1-5]\end{important}
  %\begin{warning}[by me]{My warn}\lipsum[4][1-5]\end{warning}

\end{document}
