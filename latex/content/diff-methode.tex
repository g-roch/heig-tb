\documentclass[../report]{subfiles}
\begin{document}

\newcommand\tabcritere[8]{
  \begin{center}
    \begin{tabular}{p{8em}|c|p{28em}}
      \hline
      Critère & Évaluation & Raison \\
      \hline
      \hline
      Avis des perdants & #1 & #2 \\
      \hline
      Indépendance aux petits candidats & #3 & #4 \\
      \hline
      Quasi unanimité du vote & #5 & #6 \\
      \hline
      Manipulabilité & #7 & #8 \\
      \hline
    \end{tabular}
  \end{center}
}



\part{Analyse des scrutins}

\chapter[Vote électronique: bonne ou mauvaise idée ?]{Vote électronique:\\Bonne ou mauvaise idée ?}

\section{Argumentaire politique}

Dans cette section sont présentés un certain nombre d'arguments pour ou contre le vote 
électronique (utilisé lors de vote d'État). 
Cette liste n'a pas pour vocation d'être exhaustive, mais de permettre d'avoir une idée
des implications techniques des principaux arguments \cite{olivier_elections_2022,jaberg_10_nodate}.
Les réponses qui sont faites à ces arguments sont techniques et n'ont pas pour objectif de dire si 
le vote électronique est ou non une bonne chose pour la population.

\newpage
\begin{multicols}{2}
\subsection{Pour}\label{sec:diff-meth:args:pour}
\begin{enumerate}
	\item facilité de vote en comparaison des contraintes du vote physique
	\item vote depuis l'étranger possible
	\item limitation de transmission de maladie (telle que le COVID)
	\item augmentation de la participation
%TODO reformuler ligne  "il y a également des risques avec le vote physique" ??
	\item il y a également des risques avec le vote physique
	\item résultat rapide après la clôture
\end{enumerate}
\newcolumn
\subsection{Contre}\label{sec:diff-meth:args:contre}
\begin{enumerate}
	\item trop cher à mettre en place et à maintenir
	\item compréhension impossible par certains citoyens et citoyennes
	\item recomptage impossible
	\item coercition du votant
	\item identification du votant difficilement compatible avec le secret du vote
	\item confidentialité difficile voire impossible
	\item vulnérabilité permettant la manipulabilité du résultat
%TODO reformuler ligne suivante?
	\item une panne ou vulnérabilité peut rendre le système inopérant
\end{enumerate}
\end{multicols}

\subsection{Réponses aux arguments}

\begin{description}
%TODO reformuler? OLD=Même si la facilité est relative au capacité et compétence de chacun. 
	\item[pour 1] Même si la facilité est relative aux capacités et compétences de chacun. 
		Un système facile d'utilisation qui authentifie le votant uniquement avec une 
		paire nom d'utilisateur, mot de passe et le vérifie sur un serveur central sera 
		peut-être simple à utiliser, mais est-ce que cette simplicité permet de garantir 
		les besoins de sécurité ?
	\item[pour 2 et 3] Il y a sûrement des contraintes de mise en application, mais cela simplifie 
	 	le vote pour les personnes ne pouvant ou ne voulant pas se rendre dans un bureau de vote.
 	\item[pour 4] Peut-être ou peut-être pas, l'auteure ne s'estime pas compétente pour se déterminer sur ce point.
 	\item[pour 5] Des risques existent aussi avec le vote physique, mais pour de grandes fraudes, il 
 		faut généralement la collaboration d'un nombre de personnes proportionnel à la taille de la fraude.
 		Avec le vote électronique il n'y a plus les contraintes physiques limitant la taille des fraudes.
 	\item[pour 6] Oui, les résultats peuvent être disponibles presque immédiatement.
 	\item[contre 1] C'est possible, mais il faut également mettre en perspective les bénéfices, ce qui est une question purement politique.
 	\item[contre 2] Effectivement la majorité des citoyens n'ont pas les compétences pour comprendre 
 		les protocoles de vote électronique et savoir s'ils sont acceptables.
 	\item[contre 3] Cela dépend de la solution retenue et du protocole.
 	\item[contre 4] Il y a effectivement une difficulté technique pour que le votant ne puisse pas
 		prouver ce qu'il a voté et donc limiter les contraintes qu'il pourrait subir.
 	\item[contre 5] Cela n'est pas forcement facile à mettre en place, mais le protocole mis en place 
 		pour ce rapport le permet (même si cela rajoute des contraintes).
 	\item[contre 6] Le respect de la confidentialité du vote est difficile, notamment si le terminal du votant est compromis.
 	\item[contre 7 et 8] Tout système informatique doit être considéré comme potentiellement vulnérable à des attaques.
 		Des pannes (involontaires ou consécutives à une attaque) vont sûrement se produire un jour ou l'autre; une solution de backup
 		doit donc être prête ou un report doit être possible. Ce qui serait très problématique c'est que rien ne soit fait 
 		et que l'ensemble des citoyens n'ai pas pu voter s'ils le désiraient ou qu'il y ait un report mais qu'une partie des 
 		résultats soit quand même publiés (influence du second scrutin).
\end{description}

\section{Cas de mauvaise utilisation}

Au vu des points ci-dessus, il y a des cas où l'utilisation de l'e-voting est à proscrire. 
Dans certains cas, même le vote par correspondance peut être problématique. La France ne permet que 
le vote au sein d'un bureau de vote, y compris pour les français de l'étranger. La Suisse cependant 
permet pour toutes les votations populaires le vote par correspondance.

\subsection{Votation ou élection politique}

Les élections politiques déterminent par qui le pays sera dirigé pour les années à venir. 
L'importance de ces élections est grande et donc des petits groupes de personnes peuvent 
avoir un intérêt prépondérant dans l'un ou l'autre des résultats possibles. 
Le risque d'une fraude ou tentative de fraude est présent, cependant le vote électronique 
rajoute une couche de fragilité car tout système informatique est faillible.
Le vote électronique fragilise le scrutin.

En cas de fraude avérée ou suspectée, les perdants remettront certainement le scrutin en doute, 
cependant contrairement au vote traditionnel, la vérification du résultat ne pourra se faire que 
par des experts et peut prendre beaucoup de temps (s'il faut auditer les logiciels utilisés), alors que 
pour le vote traditionnel toute personne de la population a la capacité de vérifier (au moins en partie)
qu'il n'y a pas eu de fraude, en recomptant les bulletins et vérifiant que chaque personne inscrite comme ayant
voté a effectivement voté.

Pour une fraude de grande envergure, avec le vote traditionnel il faut mettre en place un grande infrastructure, des
personne dans les bureaux de vote, du matériel de vote ou du personnel dérobant les autorisations ou les lettres pour le vote.
Ces infrastructures, d'autant plus grandes que la fraude est de grande envergure, augmentent par leur taille les chances/risques qu'elles 
soient détectées et potentiellement empêchées. Lors d'un vote électronique, selon la vulnérabilité exploitée, une poignée
de personne dans leur coin peuvent effectuer une fraude sur l'ensemble du pays.
Un autre pays, une grande entreprise ou une personne riche ont parfaitement les moyens financiers de louer les services
d'un groupe de pirates informatiques pour effectuer ou tenter de manipuler les élections.

Les votations politiques (référendum ou initiative populaire) sont tout autant voir davantage sujettes à ces risques de fraude 
car leur résultat est appliqué dans la loi directement, alors que lors d'une élection, si l'élu fait des choses qui ne
plaisent pas à la population, il y a toujours la possibilité de manifester contre (ou dans certains pays, des outils politiques
permettent à la population de montrer son désaccord).

Pour résumer, le vote électronique augmente les risques en réduisant la résilience du système à la fraude. 
Il n'est pas non-plus compris par la population général. Pour ces raisons, son utilisation est déconseillée dans ce cadre.

\subsection{Votation d'importance stratégique}

Il y a des cas où le 
recours à un vote électronique est risqué.
Une organisation qui utilise régulièrement de l'e-voting peut limiter ou interdire
son utilisation lors de votations qui concernent des décisions d'importance ou lors desquelles le risque de fraude est trop élevé.
Les critères à utiliser dépendent de l'organisation et de son environnement, voici une liste de critères possibles:
\begin{itemize}
	\item lorsqu'un opposant fort s'est prononcé (publiquement ou en privé) contre ou pour une décision
	\item lorsque la décision peut faire gagner ou perdre des sommes importantes d'argent à un autre acteur
	\item quand il y a une défiance envers une grande société ou un état
	\item lorsque l'organisation est particulièrement critiquée ou remis en cause
	\item si le report de la votation est trop compliqué ou impossible
	\item \dots
\end{itemize}

Chaque organisation choisissant d'avoir recours au vote électronique devrait mener une réflexion sur quelle
votation est trop risquée pour être effectuée en e-voting, et définir d'avance des critères clairs et précis 
concernant les possibilités et les restrictions.

\begin{nota}{Cas d'utilisation}
%TODO le paragraphe suivant n'est pas hyper clair...
	Seul des cas de non-utilisation ont été présentés dans ce chapitre. Les cas d'utilisation (spécifiques 
	aux choix effectués dans ce travail de Bachelor) sont décrits dans le \aaref{chap:usecase}.
\end{nota}

  
  \chapter{Critère de comparaison}
  
  \begin{important}{Choix des critères}
  	Le choix des critères de comparaison des différentes méthodes de vote est une question 
  	éminemment politique.
%TODO Vérifier que je n'ai pas changé le sens:
%  	Cela a la même importance que de répondre à la question quel candidat on souhaite élire en fonction
  	C'est équivalent à répondre à la question quel candidat on souhaite élire en fonction
  	des votes des électeurs. Cependant on veillera à ne pas prendre en compte les positions 
  	politiques des différents candidats.
  	On veut savoir quel candidat devrait être élu, et non comment faire pour que notre candidat 
  	favori l'emporte face aux autres candidats.
  \end{important}


  \section{Avis des perdants}

  Avec certaines méthodes, un candidat peut être élu alors que 49\% du corps électoral est
  absolument contre, et ce même lorsqu'un autre candidat avec un score plus faible n'a aucun
  électeur contre.
  Un système de choix performant devrait prendre en compte les votes qui ne sont pas pour le candidat
  élu, afin de trouver le candidat qui est satisfaisant pour le plus de monde possible.
  Le tableau \ref{fig:critere:perdant:example}, montre une possibilité de vote où une partie importante de la population
  est très mécontente du résultat.

  \begin{nota}{Scrutin d'exemple}
	Pour illustrer les critères, des simulations de scrutin sont présentées, cependant cela ne 
	veut pas dire que le type de scrutin sélectionné est le seul où ce genre de comportement 
	est présent.
  \end{nota}

	\begin{table}[h]
		\begin{center}
			\caption{Exemple de cas où une grande partie de la population est mécontente}%
			\label{fig:critere:perdant:example}%
			\adjustbox{valign=t}{%
				\begin{subtable}[h]{0.45\textwidth}
					\centering
					\caption{Préférences des électeurs}%
					\label{fig:critere:perdant:example:pref}
					\inputMake{_dyn/scrutin/critere-A.votes.tex}
				\end{subtable}
			}
			\adjustbox{valign=t}{%
				\begin{subtable}[h]{0.45\textwidth}
					\centering
					\caption{Résultats du scrutin à 1 tour}%
					\label{fig:critere:perdant:example:result}
					\inputMake{_dyn/scrutin/critere-A.maj1.tex}
				\end{subtable}
			}
		\end{center}
	\end{table}
  
  \section{Indépendance à la présence de candidat perdant}

  On peut souhaiter que si une option non choisie est retirée du scrutin, l'option vainqueure reste
  la même.
  Cependant même si ce critère n'est pas totalement respecté pour des candidats avec de très bons 
  scores, le retrait d'un petit candidat (avec un score faible) ne doit pas faire changer 
  le vainqueur.
%TODO La candidate élue est(?) passe donc(?) de A à B. ==> REFORMULER.  
  Les tableaux \aref{fig:critere:indep:example} présentent un scrutin à 3 candidats, et le même 
  scrutin, mais avec le candidat C s'étant retiré. Le candidat élu passe donc de A à B.

  \subsection{Théorème d'impossibilité d'Arrow}

  Ce théorème nous dit que dans un scrutin où les électeurs indiquent leurs préférences en classant les
  options les unes par rapport aux autres.
%TODO relire phrase suivante...
  Il n'existe pas de processus de choix indiscutable permettant un choix 
  cohérent dans le sens où le retrait d'un candidat non élu ne change pas le nom de l'élu.

  Dans les types de scrutins étudiés dans ce document, seuls la méthode Borda et le 
  jugement majoritaire échappent à ce théorème, car ils n'utilisent pas de classement des 
  options par les électeurs.
  
  \begin{table}[h]
  	\begin{center}
  		\caption{Exemple de retrait d'un petit candidat, changeant le résultat final}%
  		\label{fig:critere:indep:example}
  		\adjustbox{valign=t}{%
  			\begin{subtable}[h]{0.45\textwidth}
  				\centering
  				\caption{Préférences des électeurs}%
  				\label{fig:critere:perdant:indep:pref1}
  				\inputMake{_dyn/scrutin/critere-B1.votes.tex}
  			\end{subtable}
  		}
  		\adjustbox{valign=t}{%
  			\begin{subtable}[h]{0.45\textwidth}
  				\centering
  				\caption{Résultats du scrutin à 1 tour}%
  				\label{fig:critere:perdant:indep:result1}
  				\inputMake{_dyn/scrutin/critere-B1.maj1.tex}
  			\end{subtable}
  		}\\[1em]
	  	\adjustbox{valign=t}{%
	  		\begin{subtable}[h]{0.45\textwidth}
	  			\centering
	  			\caption{Préférences des électeurs}%
	  			\label{fig:critere:perdant:indep:pref2}
	  			\inputMake{_dyn/scrutin/critere-B2.votes.tex}
	  		\end{subtable}
	  	}
	  	\adjustbox{valign=t}{%
	  		\begin{subtable}[h]{0.45\textwidth}
	  			\centering
	  			\caption{Résultats du scrutin à 1 tour}%
	  			\label{fig:critere:perdant:indep:result2}
	  			\inputMake{_dyn/scrutin/critere-B2.maj1.tex}
	  		\end{subtable}
	  	}
  	\end{center}
  \end{table}

  \section{Quasi-unanimité du vote}

  La quasi-unanimité du vote, c'est lorsque presque l'ensemble des électeurs préfèrent 
  une option (A) à une autre option (B), alors cette deuxième option (B) n'est jamais élue/choisie.
  Il y a différentes méthodes de vote qui ne respectent pas la quasi-unanimité du vote.

%  \todo{Théorème: Borda ?}
  
  \section{Manipulabilité et vote utile}

  \subsection{Vote utile}%
  \label{diff:comp:util:util}

  Le vote utile est le fait de changer son ordre de préférence entre ses candidats favoris pour
  permettre à l'un de ses candidats favoris de remporter l'élection. 
  Je choisis de ne pas considérer comme vote utile le fait de voter pour un candidat que l'on ne
  souhaite pas voir élu, dans le but que son candidat favori soit élu. J'ai classé cette pratique 
  dans la manipulabilité du résultat (voir \aref{diff:comp:util:manip}).

  Dans le cadre de cette définition, le vote utile n'est en soi pas mauvais.
  Il permet d'aider à la recherche d'un consensus en impliquant les citoyens dans la réflexion
  et donc sans se baser exclusivement sur des mathématiques que l'ensemble de la population n'est pas forcément
  en mesure de comprendre.

  % TODO Reformuler ↓
  Les tableaux \aref{fig:critere:utile:example} montrent qu'un petit groupe de personne (3\% ou $\frac{1}{3}$ des 
  électeurs de C) en choisissant de voter utile et de classer B en premier, sont parvenus à ce que soit élu un candidat qu'ils
  avaient envie de voir au pouvoir.
  Sans ce vote utile, ces mêmes électeurs aurait vu élu un candidat qu'ils détestent.

  \begin{table}[h]
	\begin{center}
		\caption{Exemple de vote utile}%
		\label{fig:critere:utile:example}
		\adjustbox{valign=t}{%
			\begin{subtable}[h]{0.45\textwidth}
				\centering
				\caption{Préférences des électeurs}%
				\label{fig:critere:utile:example:pref1}
				\inputMake{_dyn/scrutin/critere-C1.votes.tex}
			\end{subtable}
		}
		\adjustbox{valign=t}{%
			\begin{subtable}[h]{0.45\textwidth}
				\centering
				\caption{Résultats du scrutin à 1 tour}%
				\label{fig:critere:utile:example:result1}
				\inputMake{_dyn/scrutin/critere-C1.maj1.tex}
			\end{subtable}
		}\\[1em]
		\adjustbox{valign=t}{%
			\begin{subtable}[h]{0.45\textwidth}
				\centering
				\caption{Préférences des électeurs}%
				\label{fig:critere:utile:example:pref2}
				\inputMake{_dyn/scrutin/critere-C2.votes.tex}
			\end{subtable}
		}
		\adjustbox{valign=t}{%
			\begin{subtable}[h]{0.45\textwidth}
				\centering
				\caption{Résultats du scrutin à 1 tour}%
				\label{fig:critere:utile:example:result2}
				\inputMake{_dyn/scrutin/critere-C2.maj1.tex}
			\end{subtable}
		}
	\end{center}
  \end{table}

  %\begin{nota}[élection présidentiel française 2022]{Exemple de vote utile}
  %  Par exemple lors de l'élection présidentiel française de 2022, certaines personnes de gauche
  %  radical ont choisi à voter pour J.-L. Mélanchon au premier tour alors même qu'ils aurait 
  %  plutôt voulu voter pour des candidats plus à gauche (P. Poutou).
  %  Ce n'est pas forcement une mauvais chose, car leur objectif en changeant leur vote n'est pas
  %  de que P. Poutou gagne, mais bel et bien que J.-L. Mélanchon atteigne le second tour.
  %  \todo{compléter example manipulation}
  %\end{nota}

  \subsection{Manipulabilité du résultat et corruption}%
  \label{diff:comp:util:manip}%
  \label{diff:comp:util:corruption}

  La manipulation du vote est le fait de ne pas voter pour ses candidats/options favoris, mais 
  pour un candidat que l'on ne veut pas voir au pouvoir, ceci dans l'objectif de faire en sorte que 
  notre candidat favori soit finalement élu.
  
  Les tableaux \aref{fig:critere:manip:example} montrent comment un petit nombre d'électeurs 
  (10\%) donnent leur voix à un candidat qu'ils ne veulent pas voir élu, pour réussir à faire 
  élire leur candidat favori.

  \begin{table}[h]
	\begin{center}
		\caption{Exemple de manipulation du vote}%
		\label{fig:critere:manip:example}
		\adjustbox{valign=t}{%
			\begin{subtable}[h]{0.45\textwidth}
				\centering
				\caption{Préférences des électeurs}%
				\label{fig:critere:manip:example:pref1}
				\inputMake{_dyn/scrutin/critere-D1.votes.tex}
			\end{subtable}
		}
		\adjustbox{valign=t}{%
			\begin{subtable}[h]{0.45\textwidth}
				\centering
				\caption{Résultats du scrutin à 2 tour}%
				\label{fig:critere:manip:example:result1}
				\inputMake{_dyn/scrutin/critere-D1.alternatif.tex}
			\end{subtable}
		}\\[1em]
		\adjustbox{valign=t}{%
			\begin{subtable}[h]{0.45\textwidth}
				\centering
				\caption{Préférences des électeurs}%
				\label{fig:critere:manip:example:pref2}
				\inputMake{_dyn/scrutin/critere-D2.votes.tex}
			\end{subtable}
		}
		\adjustbox{valign=t}{%
			\begin{subtable}[h]{0.45\textwidth}
				\centering
				\caption{Résultats du scrutin à 2 tour}%
				\label{fig:critere:manip:example:result2}
				\inputMake{_dyn/scrutin/critere-D2.alternatif.tex}
			\end{subtable}
		}
	\end{center}
  \end{table}

  La corruption du scrutin correspond à l'achat\footnote{Par achat on entend
  tout échange visant à limiter le choix libre du vote allant d'un petit avantage à la menace 
  de mort sur le/la votant·e ou ses proches.} de leur vote auprès des électeurs.
  La corruption est légèrement différente de la manipulation, dans le sens où le candidat
  que l'on cherche à élire à la fin n'est pas forcément celui qui est préféré par le votant.
  Tous les systèmes de vote sont vulnérables à ce genre de manipulation, mais un bon système de 
  vote devrait nécessiter une très forte corruption pour que le résultat final soit impacté.
  Certaines techniques peuvent être mises en place pour limiter la corruption des votants (secret
  du vote, impossibilité de prouver pour qui on a voté, liberté de la presse …), mais 
  ces techniques ne peuvent pas garantir l'absence de corruption.
  
  Les tableaux \aref{fig:critere:corruption:example} montrent comment le fait de corrompre un petit nombre d'électeur 
  (3\%) permet de faire élire le candidat A alors que les électeurs corrompus ne voulait pas réellement voir le dit candidat A élu.
  
  
  \begin{table}[h]
  	\begin{center}
  		\caption{Exemple de corruption du vote}%
  		\label{fig:critere:corruption:example}
  		\adjustbox{valign=t}{%
  			\begin{subtable}[h]{0.45\textwidth}
  				\centering
  				\caption{Préférences des électeurs}%
  				\label{fig:critere:corruption:example:pref1}
  				\inputMake{_dyn/scrutin/critere-E1.votes.tex}
  			\end{subtable}
  		}
  		\adjustbox{valign=t}{%
  			\begin{subtable}[h]{0.45\textwidth}
  				\centering
  				\caption{Résultats du scrutin à 2 tour}%
  				\label{fig:critere:corruption:example:result1}
  				\inputMake{_dyn/scrutin/critere-E1.alternatif.tex}
  			\end{subtable}
  		}\\[1em]
  		\adjustbox{valign=t}{%
  			\begin{subtable}[h]{0.45\textwidth}
  				\centering
  				\caption{Préférences des électeurs}%
  				\label{fig:critere:corruption:example:pref2}
  				\inputMake{_dyn/scrutin/critere-E2.votes.tex}
  			\end{subtable}
  		}
  		\adjustbox{valign=t}{%
  			\begin{subtable}[h]{0.45\textwidth}
  				\centering
  				\caption{Résultats du scrutin à 2 tour}%
  				\label{fig:critere:corruption:example:result2}
  				\inputMake{_dyn/scrutin/critere-E2.alternatif.tex}
  			\end{subtable}
  		}
  	\end{center}
  \end{table}
  
  \section{Autre critère non pris en compte}
%  \todo{À déplacer ou supprime (les 3 subsections qui suivent)}
%  \subsection{Votes blancs, nuls et abstentions}
%
%  Les votes blancs et nuls devraient toujours être comptés, ne serait-ce que pour des raisons de
%  statistique.
%  Savoir qu'il y a eu 90\% de votes blancs permet de voir qu'il y a un problème, même
%  s'il n'y a aucun moyen de savoir ce qui s'est passé, au simple regard de cette statistique.
%
%  Dans le cas où l'on veut prendre en compte les votes blancs, nuls ou abstentions, il est primordial
%  de faire attention que cela ne privilégie pas une option par rapport à une autre.
%
%  \begin{nota}{Décompte de vote blancs non-neutre}
%    Lors d'un référendum contre une décision du gouvernement, il faut qu'il y ait plus de 50\% 
%    de vote contre pour que la décision soit révoquée.
%
%    Dans le cas où les votes blancs, nuls ou les abstentions sont pris en compte dans 
%    le nombre de personnes ayant validement voter, cela augmente le nombre de votes «~contre~» 
%    dont il y a besoin pour refuser le choix du gouvernement.
%
%    De ce fait, ces «votes» vont donc dans le sens du gouvernement et ne sont donc pas neutres. 
%
%    Si les votes blancs/nuls font juste baisser le quorum, et que si le quorum n'est pas atteint
%%TODO relire encore le paragraphe suivant.
%    une seconde votation a lieu, alors, c'est un usage plus correct de ces voix.
%    %Une manière de faire qui semble être plus correcte concernant ces voix est en présence dans le cas où les votes blancs/nuls font juste baisser le quorum, et que si le quorum n'est pas atteint
%    %une seconde votation a lieu.
%  \end{nota}
%
%
%  \subsection{Corps électoral}
%
%  Pour avoir une bonne représentation de ce que veut la population, avoir un faible taux de votes
%  blancs ou nuls et d'abstentions n'est pas pas suffisant.
%  En effet dans la majorité des pays, une grande partie de la population n'a pas le droit de vote.
%
%  \begin{nota}{Droit de vote en Suisse}
%    En 2015, 37\%\footnote{au niveau fédéral, source: \url{https://www.swissinfo.ch/fre/democratiedirecte/élections-2019\_qui-peut-voter-en-suisse--et-qui-ne-peut-pas-/45264666}}
%    de la population Suisse n'avait simplement pas le droit de vote:
%    \begin{enumerate}
%      \item les étrangers comme dans beaucoup de pays.
%      \item les jeunes jusqu'à 18 ans, cet âge peut varier en fonction du pays
%      \item dans certains pays (mais pas la Suisse), les personnes incarcérées, n'ont pas 
%        le droit de vote.
%      \item et enfin toutes les personnes sous curatelle de portée générale ne peuvent 
%        également pas voter.
%      \item jusqu'à récemment certains pays\footnote{L'Arabie Saoudite a donné le droit de vote aux femmes en 2011, source: \url{https://fr.wikipedia.org/wiki/Droit\_de\_vote\_des\_femmes}} ne donnaient pas le droit de vote aux femmes.
%    \end{enumerate}
%  \end{nota}
%
%  Ces diverses restrictions font qu'une partie de la population vit dans un pays, mais n'a
%  pas la possibilité de participer aux décisions collectives.
%  Lorsqu'on essaye d'interpréter un scrutin réel, il faut donc prendre cela en compte.
%  Un candidat élu à 70\% n'est pas élu à 70\% de la population, ni même à 70\% des personnes
%  ayant le droit de vote, ni encore à 70\% des personnes inscrites sur les listes électorales,
%  ni même enfin à 70\% des votants, mais à 70\% des votants n'ayant pas voté blanc et dont leur 
%  bulletin n'a pas été considéré comme nul.
%  En fonction du pays et de la votation, cela peut avoir une grande importance pour déterminer l'opinion de la population générale.
%
%  Il faut aussi tenir compte du fait que certains pays donnent aussi le droit vote aux personnes ayant la nationalité, mais 
%  résidant à l'étranger.
%  Le résultat ne correspond donc pas à ce que veulent les résidants, mais est influencé aussi par ce
%  que veulent des personnes vivantes à l'étranger.
%
%  \begin{table}[h]
%    \caption{Suisse: Initiative populaire «~99~\%~»}%
%    \label{tab:votation:99pourcent}%
%    % voir : https://www.bfs.admin.ch/bfs/fr/home/statistiques/politique/votations/annee-2021/2021-09-26/initiative-99-pour-cent.html
%    % voir : https://www.bfs.admin.ch/bfs/fr/home/statistiques/population/effectif-evolution/population.html
%    \begin{center}
%      \begin{tabular}{lrccc}
%        \hline
%        & & Électeur & Pour [\%] & Contre [\%] \\
%        \hline
%        \hline
%        Résultat officiel        &              & 2~810~307 & 35,12          & 64,88          \\
%        Avec les votes blancs    & $+58~499$    & 2~868~806 & 34,41 -- 36,45 & 63,55 -- 65,59 \\
%        Avec les votes nuls      & $+14~073$    & 2~882~879 & 34,24 -- 36,76 & 63,24 -- 65,76 \\
%        Sans les abstentions     & $+2~636~289$ & 5~519~198 & 17,88 -- 66,96 & 33,04 -- 82,12 \\
%        Populatation tot. (2020) & $+3~151~132$ & 8~670~300 & 11,38 -- 78,97 & 21,03 -- 88,62 \\
%        \hline
%      \end{tabular}
%    \end{center}
%  \end{table}
%
%  La Suisse a voté le 26 septembre 2021 sur l'initiative «~99~\%~».
%  Cette initiative a été largement refusée à plus de 64\%.
%%TODO reformuler OLD="montre les pourcentages de la population possible"...
%  Le \aaref{tab:votation:99pourcent} montre les pourcentages de la population possible dans le 
%  cas où personne n'aurait voté blanc, aucun bulletin nul décompté, sans abstention et 
%  finalement si l'ensemble de la population avait le droit de vote et votait.
%  Par exemple on sait avec certitude que 21,03\% de la population était contre 
%  (l'initiative a été rejetée), mais la proportion de personnes qui sont contre peut varier entre 
%  21,03\% et 88,62\% si l'on ne regarde que le résultat du scrutin.
%  La population prise en compte et celle de 2020 (une année avant le scrutin).
%
%  \subsection{Réflexion et débat}
%
%  Certains scrutins se déroulent en plusieurs tours, il peut être intéressant de demander à la 
%  population de classer tous les candidats/options et de simuler tous les tours.
%  Cette technique permet aux électeurs de ne se déplacer qu'une seule fois aux urnes.
%  Malheureusement, en simulant les tours, on supprime également la possibilité d'avoir un
%  débat et donc que les électeurs puissent changer d'avis.
%  Les compromis entre candidats sont également impossibles.
%
%  Un scrutin à plusieurs tours réels permet d'améliorer le débat politique, même si un scrutin 
%%TODO "certainement plus" ????  
%  à 10 tours lasserait certainement plus que le bénéfice possible.

  \section{Évaluation}
  
  Dans le \aaref{tab:criteres:scrutin} est présentée pour chaque critère d'évaluation, les conditions d'obtention pour chaque évaluation.
  Ces conditions d'obtention sont des choix personnels dans l'objectif que le scrutin choisi permette de se rapprocher aux mieux de l'unanimité.

  \begin{table}
  \caption{Critères d'évaluations des scrutins}%
  \label{tab:criteres:scrutin}%
  \begin{center}
    \begin{tabular}{p{8em}|c|p{28em}}
    %\begin{tabular}{l|c|p{20em}}
      \hline
      Critère & Évaluation & Raison \\
      \hline
      \hline
      Avis des perdants & \cellcolor{green}bien & Le vote de chacun est pris en compte pour chacun des candidats. \\
      \cline{2-3}       & \cellcolor{green!25!yellow}correct & Si le vote de chacun est pris en compte, les électeurs n'ont cependant pas forcement la possibilité d'exprimer toute la nuance voulue. \\
      \cline{2-3}       & \cellcolor{orange}moyen & De manière générale le vote de chacun est pris en compte. Il existe cependant des cas particuliers plausibles où l'avis de 49\% de la population n'est pas pris en compte. \\
      \cline{2-3}       & \cellcolor{red}mauvais & L'avis de 49\% de la population n'a pas d’influence sur le résultat. Ces 49\% peuvent changer leur votes comme ils le souhaitent, le candidat élu, reste élu, et conserve le même score. \\
      \cline{2-3}       & \cellcolor{red}très mauvais & Plus de 50\% de la population peuvent être contre le candidat élu. \\
      \hline
%TODO "indépendant AUX candidats non-élus..." quid?      
      Indépendance aux petits candidats & \cellcolor{green}bien & Le résultat est totalement indépendant aux candidats non-élus. \\
      \cline{2-3}
                       & \cellcolor{orange}moyen & La présence d'un petit candidat favorise les candidats proches politiquement. \\
      \cline{2-3}
%TODO reformuler ligne suivante. (partie droite)
                       & \cellcolor{red}mauvais &  La présence d'un petit candidat \textbf{défavorise} les candidats proches politiquement. Moins bien que s'ils les favorisaient, car cela pousse à ne pas proposer de nuance politique qui peut être essentielle pour certaines parties de la population. \\
      \hline
      Quasi unanimité du vote  & \cellcolor{green}bien & Le cas où quasiment la totalité de la population ($\frac{3}{4}$) préfère un autre candidat au candidat élu n'est pas possible. \\
      \cline{2-3}
                       & \cellcolor{orange}moyen & Pas de garantie, cependant le scrutin limite la proportion de la population très mécontente du résultat. \\
      \cline{2-3}
                       & \cellcolor{red}mauvais & Plus de $\frac{3}{4}$ de la population peuvent préférer un autre candidat dans un duel. \\
      \hline
      Manipulabilité & \cellcolor{green}bien & La meilleure technique pour voir son ou ses favori élu-s consiste à voter selon ses vraies préférences. \\
      \cline{2-3}
                     & \cellcolor{green!25!yellow}correct & Le meilleure technique consiste à donner ses vraies préférences, même si dans certain cas spécifiques (pour certains votants uniquement), une meilleures stratégie peux exister. \\
      \cline{2-3}
                     & \cellcolor{orange}moyen & La meilleure technique consiste à voter différemment de ses préférences pour donner plus de chance à son ou ses candidats favoris. \\
      \cline{2-3}
                     & \cellcolor{red}mauvais & La meilleure technique de vote consiste à se baser sur les \textbf{sondages} pour donner une voix à un candidat ayant une chance de l'emporter et contre lequel on est pas trop opposé. \\
      \hline
    \end{tabular}
  \end{center}
  \end{table}

  \chapter{Les scrutins}
  
  Les simulations de scrutins on été réalises grâce à du code écrit pour l'occasion disponible 
  à l'adresse \url{https://github.com/g-roch/heig-tb-report/tree/main/simulation-scrutin}

%TODO dans la suite, 50% +1 c'est la majorité ABSOLUE
%TODO dans la suite, les calculs 3.5 et 4.5 ne doivent-ils pas être revus?
  \begin{important}[$50\% + 1$ versus $ > 50\%$]{Définition de la majorité}
    On trouve fréquemment l'expression $50\%+1$ pour désigner le nombre de votes requis pour
    obtenir la majorité, alors qu'il faut juste plus de $50\%$ des voix pour la majorité.
    Même si cette distinction parait minime, pour des scrutins avec un nombre d'électeurs 
    impair c'est un point important.

    Si A a 4 voix, B a 1 voix et C a 2 voix, on a un total de 7 électeurs.
    Dans un cas pour avoir la majorité nous devons obtenir le nombre de voix suivant
    \[ \text{plus que }50\%\text{ de }7 = \frac{7}{2} = 3.5 \]
    alors que dans l'autre cas
    \[ \text{au moins }(50\%\text{ de }7) + 1 = \frac{7}{2} + 1 = 4.5 \]

    Nous avons donc un cas où il faut un score $ > 3.5 $ (4 voix au minimum) et un autre cas où il faut un score 
    $ \geqslant 4.5 $ (5 voix au minimum!)
    Le candidat A se retrouve donc élu avec une définition et non élu avec l'autre.

    Dans ce document et le code associé seule la définition $> 50\%$ a été utilisée
  \end{important}

  \section{Scrutin proportionnel plurinominal}

  Les électeurs votent pour une liste (de candidats) de leur choix.
  Chaque liste remporte un nombre de sièges proportionnel au nombre de voix reçues.
  Les sièges sont finalement distribués dans chaque liste soit par ordre de présence dans
  la liste, soit en fonction du nombre de voix de chaque candidat dans la liste.

  \paragraph{Scrutin avec seuil}

  Lors de certaines élections, il peut y avoir un seuil à atteindre pour qu'une liste puisse 
  prétendre à un siège.
  C'est-à-dire qu'il faut avoir par exemple les voix nécessaires pour 3 sièges pour permettre
  à la liste de rentrer dans le groupe de liste recevant des sièges.
  S'il n'y a pas de seuil, chaque liste avec un nombre de voix suffisant pour 1 siège participe 
  à la répartition.
  
  \paragraph{Préférentiel ou listes bloquées}

  En listes bloquées, les sièges sont répartis au sein des listes par ordre d'apparition 
  des candidats.
  En préférentiel, les sièges sont répartis en fonction des préférences des électeurs.
  Ces derniers peuvent changer l'ordre des candidats, biffant/rajoutant certain ou même
  en cumulant certains candidats plusieurs fois.
  
  \paragraph{Répartition des sièges}

  Les candidats ne pouvant pas être élus à moitié, diverses méthodes existent pour répartir
  les sièges entre les diverses listes.
  Chaque méthode peut donner des résultats différents.
  
  \section{Scrutin uninominal majoritaire}\label{diff:scrutin-majoritaire}
  
  Chaque votant vote pour son option préférée.
  Pour chaque option on compte le nombre d'électeurs l'ayant choisi, l'option avec le plus 
  d'électeurs l'emporte.
  
  \subsection{Majorité relative ou absolue}
  \subsubsection{Majorité absolue}
  Dans le cas d'un scrutin à majorité absolue, l'option avec plus de 50\% de vote est déclarée
  vainqueure.
  Dans le cas où aucune option n'atteint les 50\%, un tour supplémentaire est généralement
  organisé.
  Ce tour supplémentaire peut être à la majorité relative dans une volonté de limiter le nombre
  de tours.
  
  \begin{nota}{Pourquoi 50\%~?}
    Pour certains scrutins plus importants, le seuil est plus élevé que 50\%, par exemple 60\% ou 70\%. 
    En quoi est-ce que 50\% est un bon choix, est-ce que ça veut dire que le choix de la moitié
    des votants n'a pas importance~? 
  \end{nota}
  
  \subsubsection{Majorité relative}
  Dans le cas d'un scrutin à majorité relative, l'option avec le plus de voix est choisie, 
  quel que soit son pourcentage de voix. Il n'y a donc pas besoin d'un second tour s'il n'y 
  a pas d'ex aequo. Ce scrutin est décrit dans l'algorithme \aref{scrutin:maj-uni-1t}.

  \begin{algorithm}
    \caption{Scrutin majoritaire uninominal à 1 tour}%
    \label{scrutin:maj-uni-1t}
    \begin{algorithmic}[1]
      \REQUIRE{$votes$[n° du votant] = n° du choix favori}
      \ENSURE{Liste des n° des choix vainqueurs}
      \FORALL{$vote \leftarrow votes$}
      \STATE{$total[vote]$++}
      \ENDFOR{}
      \RETURN{Les indexes de max$(total)$}
    \end{algorithmic}
  \end{algorithm}


  \subsection{Nombre de tours}

  Lors d'un scrutin à la majorité relative, un seul tour est nécessaire.
  Pour la variante à la majorité absolue, un candidat n'est pas forcément élu lors du
  premier tour. Dans ce dernier cas, un tour supplémentaire doit être organisé, les options
  les plus connues sont~: 
  \begin{itemize}
    \item Garder les 2 meilleurs candidats. C'est le dernier tour, il y en a forcément un 
      qui aura plus de 50\%\footnote{S'il n'y a pas ex aequo et que les votes blancs/nuls 
      et abstentions n'influencent pas le résultat}. C'est ce qui est utilisé pour l'élection
      présidentielle française. Ce scrutin est décrit dans l'algorithme \aref{scrutin:maj-uni-2t}.
    \item Éliminer le pire des candidats, s'il y a encore plus que 2 candidats, il peut avoir
      encore d'autres tours. C'est ce qui est utilisé pour l'élection de chaque membre du
      conseil fédéral à partir du 3\up{e} tours.
  \end{itemize}

  Entre chaque tour les candidats ont habituellement la possibilité de se retirer du scrutin.
  
  \begin{nota}{Qui éliminer~?}
    Habituellement, les candidats/options avec le moins de voix sont éliminés, mais
    les électeurs de ces candidats sont-ils vraiment ceux qui sont le plus susceptibles de changer d'avis~?
    Est-ce qu'un électeur d'un des 2 candidats proche mutuellement n'aurait pas plus
    de facilité à changer son vote~?
  \end{nota}

  \begin{algorithm}
    \caption{Scrutin majoritaire uninominal à 2 tour}%
    \label{scrutin:maj-uni-2t}
    \begin{algorithmic}[1]
      \REQUIRE{$votes$[n° du votant] = Liste des choix par ordre de préférence décroisante}
      \ENSURE{Liste des n° des choix vainqueurs}
      \STATE{$majorité \leftarrow \frac{\text{len}(votes)}{2} $}
      \FORALL{$vote \leftarrow votes$}
      \STATE{$choix \leftarrow vote[0]$ \COMMENT{Premier choix du votant}}
      \STATE{$total[choix]$++}
      \ENDFOR{}
      \IF{$\text{max}(total) > majorité$}
      \RETURN{Index de max$(total)$}
      \ENDIF{}
      \STATE{}
      \STATE{$restant \leftarrow $ index des 2 valeurs maximal de $total$}
      \STATE{vider $total$}
      \FORALL{$vote \leftarrow votes$}
      \STATE{$choix \leftarrow $ première valeur de $vote$ se trouvant dans $restant$}
      \STATE{$total[choix]++$}
      \ENDFOR{}
      \IF{$\text{max}(total) > majorité$}
      \RETURN{Index de max$(total)$}
      \ELSE{}
      \STATE{\COMMENT{Les 2 choix restants sont ex-aequo}}
      \RETURN{$restant$}
      \ENDIF{}
    \end{algorithmic}
  \end{algorithm}

  \subsection{Élection étrange}

  \begin{table}[h]
    \begin{center}
      \caption{Cas limites d'un scrutin à la majorité relative (à 1 tour)}%
      \label{fig:diff:maj1:caslim1}
      \adjustbox{valign=t}{%
        \begin{subtable}[h]{0.45\textwidth}
          \centering
          \caption{Préférences des électeurs}%
          \label{fig:diff:maj1:caslim1:pref}
          \inputMake{_dyn/scrutin/maj1-caslim-1.votes.tex}
        \end{subtable}
      }
      \adjustbox{valign=t}{%
        \begin{subtable}[h]{0.45\textwidth}
          \centering
          \caption{Résultats du scrutin}%
          \label{fig:diff:maj1:caslim1:result}
          \inputMake{_dyn/scrutin/maj1-caslim-1.maj1.tex}
        \end{subtable}
      }
    \end{center}
  \end{table}

  Les tableaux \aref{fig:diff:maj1:caslim1} montrent les préférences d'un corps 
  électoral et le résultat d'une votation avec un scrutin majoritaire à 1 tour.

  On peux constater plus de 60\% des électeurs (ii et iii) préfèrent B à A ou C à A. 
  Cependant comme A a fait plus de voix que B ou C, c'est donc cette option qui est élue.
  Si B ou C s'était retiré avant le scrutin ou s'il y avait eu une candidature commune, A aurait été largement battu. 

  \subsection{Évaluation}
  \subsubsection{À 2 tours}
  \tabcritere%
    {\cellcolor{red}très mauvais}{Une grande part de la population (>50\%) peut être contre l'élu·e.}%
    {\cellcolor{red}mauvais}{Les petits candidats «~volent~» des voix aux autres candidats proches d'eux politiquement.}%
    {\cellcolor{red}mauvais}{Si les voix sont suffisamment divisés dans un des camps, aucun de leurs candidats pourrait atteindre le second tour, alors que ce camp représente peut-être plus de 75\% de la population.}%
    {\cellcolor{red}mauvais}{La meilleure technique de vote consiste à se baser sur des sondages pour avoir une estimation des résultats et voter pour le candidat qui a le plus de chance d'être élu et pour lequel on n'est pas entièrement contre. Les petits candidats qui auraient une chance de l'emporter, mais non présents dans les sondages ou mal classés ne recevront que peu de voix, alors qu'ils auraient pu faire de très bons scores.}

  \subsubsection{À 1 tour}
  \tabcritere%
    {\cellcolor{red}très mauvais}{Une grande part de la population (>50\%) peut être contre l'élu·e.}%
    {\cellcolor{red}mauvais}{Les petits candidats «~volent~» des voix aux autres candidats proches d'eux politiquement.}%
    {\cellcolor{red}mauvais}{Si les voix sont suffisamment divisées dans un des camps, et que dans l'autre camp il y a un candidat unique, le résultat sera biaisé en faveur du candidat unique, alors que potentiellement plus de 75\% de le population serait pour un autre candidat, n'importe lequel.}%
    {\cellcolor{red}mauvais}{La meilleure technique de vote consiste à se baser sur des sondages pour avoir une estimation des résultats et voter pour le candidat qui a le plus de chance d'être élu et pour lequel on n'est pas entièrement contre. Les petits candidats qui auraient une chance de l'emporter, mais non présents dans les sondages ou mal classés ne recevront que peu de voix, alors qu'ils auraient pu faire de très bons scores.}

  \section{Méthode Borda}\label{sec:scrutin:borda}
  
  Chaque votant classe toutes ou une partie des options dans l'ordre de leurs préférences.
  Les candidats reçoivent des points en fonction de leur position dans l'ordre de chaque bulletin.
  Le/la vainqueur·e est celui qui a obtenu le plus de points.

  \subsection{Distribution des points}

  Dans la description de cette méthode de vote faite par J.-C. Borda, il donnait 1 point au
  candidat classé en dernier, puis un point supplémentaire pour l'avant-dernier et ainsi
  de suite jusqu'au premier candidat.
  Dans la pratique actuelle~\cite{emerson_original_2013} les points sont distribués du 
  premier au dernier $(n, n-1, …, 1)$.
  Si tous les électeurs classent l'ensemble des candidats, alors ces deux méthodes donnent le même 
  résultat. Cependant lorsqu'on autorise le classement partiel, il y a des différences.

  Pour éviter toute ambiguïté du fait de de la présence de deux méthodes Borda, 
  je désigne par \textbf{classique} la méthode initialement décrite par Borda avec distribution des points à partir du candidat classé en dernier (1 point)
  et par  \textbf{moderne} la méthode distribuant
  les points à partir du candidat classé en premier ($n$ points).

  \begin{table}
    \begin{center}
      \caption{Méthode Borda différence entre les méthodes de distribution des points}%
      \label{fig:diff:borda:caslim2}
      \adjustbox{valign=t}{
        \begin{subtable}[h]{0.35\textwidth}
          \centering
          \caption{Préférences des électeurs}
          \inputMake{_dyn/scrutin/borda-caslim-2.votes-c.tex}
        \end{subtable}
      }
      \adjustbox{valign=t}{
        \begin{subtable}[h]{0.25\textwidth}
          \centering
          \caption{Points classique}
          \inputMake{_dyn/scrutin/borda-caslim-2.bordaclasse.tex}
        \end{subtable}
      }
      \adjustbox{valign=t}{
        \begin{subtable}[h]{0.25\textwidth}
          \centering
          \caption{Points moderne}
          \inputMake{_dyn/scrutin/borda-caslim-2.bordatot.tex}
        \end{subtable}
      }
    \end{center}
  \end{table}

  Comme le montrent les tableaux \aref{fig:diff:borda:caslim2}, la méthode classique pousse 
  les votants à classer un maximum de candidats et donc aide à la recherche d'un 
  compromis convenant à tout le monde.
  Alors que la méthode moderne pousse plus à ne classer que le candidat favori, si on grande
  partie des votants choisissent de faire cela, on se retrouve dans une situation proche 
  d'un scrutin majoritaire à majorité relative.

  Les deux méthodes de distribution des points sont décrites dans les algorithmes \aref{scrutin:borda-classique} et \aref{scrutin:borda-moderne}.

  \begin{algorithm}
    \caption{Méthode Borda (avec distribution des points \textbf{classique})}%
    \label{scrutin:borda-classique}
    \begin{algorithmic}[1]
      \REQUIRE{$votes$[n° du votant] = Liste des choix par ordre de préférence décroisante}
      \ENSURE{Liste des n° des choix vainqueurs}
      \FORALL{$vote \leftarrow votes$}
      \STATE{$point \leftarrow \text{len}(vote)$}
      \FORALL{$option \leftarrow vote$}
      \STATE{$total[option] \leftarrow total[option] + point$}
      \STATE{$point--$}
      \ENDFOR{}
      \ENDFOR{}
      \RETURN{Les indexes de max$(total)$}
  \end{algorithmic}
  \end{algorithm}
  
  \begin{algorithm}
    \caption{Méthode Borda (avec distribution des points \textbf{moderne})}%
    \label{scrutin:borda-moderne}
    \begin{algorithmic}[1]
      \REQUIRE{$votes$[n° du votant] = Liste des choix par ordre de préférence décroisante}
      \ENSURE{Liste des n° des choix vainqueurs}
      \FORALL{$vote \leftarrow votes$}
      \STATE{$point \leftarrow $ Nombre d'option disponible}
      \FORALL{$option \leftarrow vote$}
      \STATE{$total[option] \leftarrow total[option] + point$}
      \STATE{$point--$}
      \ENDFOR{}
      \ENDFOR{}
      \RETURN{Les indexes de max$(total)$}
    \end{algorithmic}
  \end{algorithm}

  \subsection{Manipulabilité de l'élection}
  \begin{table}[h]
    \begin{center}
      \caption{Cas limites d'un scrutin utilisant la méthode Borda}%
      \label{fig:diff:borda:caslim1}
      \adjustbox{valign=t}{
        \begin{subtable}[h]{0.40\textwidth}
          \centering
          \caption{Préférences des électeurs}%
          \label{fig:diff:borda:caslim1:A}
          \inputMake{_dyn/scrutin/borda-caslim-1A.votes-c.tex}
        \end{subtable}
      }
      \adjustbox{valign=t}{
        \begin{subtable}[h]{0.20\textwidth}
          \centering
          \caption{Points classique}%
          \label{fig:diff:borda:caslim1:B}
          \inputMake{_dyn/scrutin/borda-caslim-1A.bordaclasse.tex}
        \end{subtable}
      }
      \adjustbox{valign=t}{
        \begin{subtable}[h]{0.20\textwidth}
          \centering
          \caption{Points moderne}%
          \label{fig:diff:borda:caslim1:C}
          \inputMake{_dyn/scrutin/borda-caslim-1A.bordatot.tex}
        \end{subtable}
      }\\[1em]
      \adjustbox{valign=t}{
        \begin{subtable}[h]{0.40\textwidth}
          \centering
          \caption{Préférences des électeurs}%
          \label{fig:diff:borda:caslim1:D}
          \inputMake{_dyn/scrutin/borda-caslim-1B.votes-c.tex}
        \end{subtable}
      }
      \adjustbox{valign=t}{
        \begin{subtable}[h]{0.20\textwidth}
          \centering
          \caption{Points classique}%
          \label{fig:diff:borda:caslim1:E}
          \inputMake{_dyn/scrutin/borda-caslim-1B.bordaclasse.tex}
        \end{subtable}
      }
      \adjustbox{valign=t}{
        \begin{subtable}[h]{0.20\textwidth}
          \centering
          \caption{Points moderne}%
          \label{fig:diff:borda:caslim1:F}
          \inputMake{_dyn/scrutin/borda-caslim-1B.bordatot.tex}
        \end{subtable}
      }
    \end{center}
  \end{table}

  Les tableaux~\ref{fig:diff:borda:caslim1:A} à~\aref{fig:diff:borda:caslim1:C} indiquent les 
  préférences des électeurs, les tableaux~\ref{fig:diff:borda:caslim1:D} 
  à~\ref{fig:diff:borda:caslim1:F} montrent la même élection avec une tentative de manipulation 
  du vote.

  Nous pouvons constater que lorsqu'on utilise le décompte de points moderne, une petite 
  proportion des électeurs (16\%) arrive à faire basculer le résultat à leur avantage.


  \subsection{Évaluation}
  \subsubsection{Borda avec distribution des points classique}
  \tabcritere%
    {\cellcolor{green}bien}{Chaque vote est pris en compte même s'il ne concerne pas le candidat élu.}%
    {\cellcolor{orange}moyen}{Les petits candidats offrent un avantage aux candidats proches. Dans le cas d'une élection équilibrée, ce n'est pas un problème.}%
    {\cellcolor{orange}moyen}{Ne respecte pas ce critère, mais essaie de minimiser le nombre de personnes très mécontentes du résultat.}%
    {\cellcolor{green}bien}{Les électeurs on tout intérêt à donner leurs vraies préférences en général}
  \subsubsection{Borda avec distribution des points moderne}
  \tabcritere%
    {\cellcolor{green}bien}{Chaque vote est pris en compte même s'il ne concerne pas le candidat élu.}%
    {\cellcolor{green}bien}{Les candidats non classés en premier n'influence pas les points donnés au premier, pour autant que le nombre total de candidat reste le même.}%
    {\cellcolor{orange}moyen}{Ne respecte pas ce critère, mais essaie de minimiser le nombre de personnes très mécontentes du résultat.}%
    {\cellcolor{orange}moyen}{Les électeurs on intérêt à classer moins bien leurs seconds choix pour avantager leur candidat favori.}


  \section{Méthode de Condorcet}

  Les électeurs classent les candidats par ordre de préférence. 
  Puis, sont simulés des duels entre chaque candidat au scrutin majoritaire.
  Le candidat ayant battu l'ensemble des autres candidats est élu.

  \subsection{Paradoxe de Condorcet}

  Le paradoxe de Condorcet, lors d'un vote par classement de 3 options (A, B, C),
  c'est qu'il peut y avoir une majorité de votants préférant C à A, une autre majorité 
  préférant B à C, et qu'une dernière majorité préfère A à B.
  Dans un tel cas, quelleque soit l'option retenue, il y a toujours plus de 50\% des votants
  qui seraient pour changer d'option.

  Dans le cas où il y a un paradoxe de Condorcet, d'autres méthodes doivent être utilisées pour désigner
  le gagnant.
  Cela peut être une autre méthode de vote ou un algorithme spécifique.
  \subsection{Élection étrange}
  \begin{table}[h]
    \begin{center}
      \caption{Scrutin à la Condorcet ne prenant pas en compte les perdants}%
      \label{fig:diff:condorcet:caslim1}
      \adjustbox{valign=t}{%
        \begin{subtable}[h]{0.45\textwidth}
          \centering
          \caption{Préférences des électeurs}%
          \label{fig:diff:condorcet:caslim1:pref}
          \inputMake{_dyn/scrutin/condorcet-caslim-1.votes-c.tex}
        \end{subtable}
      }
      \adjustbox{valign=t}{%
        \begin{subtable}[h]{0.45\textwidth}
          \centering
          \caption{Résultats du scrutin}%
          \label{fig:diff:condorcet:caslim1:result}
          \inputMake{_dyn/scrutin/condorcet-caslim-1.condorcet.tex}
        \end{subtable}
      }
    \end{center}
  \end{table}

  Les tableaux~\aref{fig:diff:condorcet:caslim1} montrent un scrutin effectué avec la méthode de Condorcet,
  donnant vainqueur A.
  L'option C a été classée en première ou deuxième position par l'ensemble des électeurs, alors que l'option
  vainqueure (A) a été refusée sèchement par un tiers des électeurs.
  Dans ce scrutin l'avis d'un tiers de la population a purement été ignoré.

  \subsection{Évaluation}
  \tabcritere%
    {\cellcolor{red}mauvais}{Seul l'avis de la majorité est pris en compte, l'avis des 49\% restant est ignoré.}%
    {\cellcolor{green}bien}{Lorsqu'il y a un unique vainqueur, la méthode est résistant aux petits candidats.}%
    {\cellcolor{green}bien}{Lorsqu'il y a un unique vainqueur, ce critère est forcement respecté de part la définition du scrutin.}%
    {\cellcolor{green}bien}{Le système de vote pousse les électeurs à donner leur vraie préférence}

  \section{Vote par approbation}

  Chaque électeur indique le ou les candidats qu'il trouve acceptable d'être élu.
  Le candidat ayant reçu le plus d'approbation est élu.
  Ce système est facile à mettre en place pour les élections actuelles, car il suffirait d'autoriser
  les électeurs à glisser plusieurs bulletins différents dans l'urne.

  \subsection{Élection étrange}

  \begin{table}[h]
    \begin{center}
      \caption{Vote par approbation: cas étrange}%
      \label{fig:diff:appro:caslim1}
      \adjustbox{valign=t}{
        \begin{subtable}[h]{0.45\textwidth}
          \centering
          \caption{Préférences des électeurs}
          \inputMake{_dyn/scrutin/apro-caslim-1.votes.tex}
        \end{subtable}
      }
      \adjustbox{valign=t}{
        \begin{subtable}[h]{0.45\textwidth}
          \centering
          \caption{Résultats}
          \inputMake{_dyn/scrutin/apro-caslim-1.apro.tex}
        \end{subtable}
      }
    \end{center}
  \end{table}

  Dans l'élection représentée dans les tableaux \aref{fig:diff:appro:caslim1}, l'option
  C gagne alors que 75\% de l'électorat lui préfère largement A et 75\% lui préfère B.

  \subsection{Évaluation}
  \tabcritere%
    {\cellcolor{green!25!yellow}correct}{L'avis des perdants est pris en compte, cependant on ne peut indiquer que l'on ne veut pas de 2 candidats, mais que malgré tout, on est moins défavorable à l'un des deux.}%
    {\cellcolor{green}bien}{Chaque candidat a un score propre, ne dépendant pas des autres candidats.}%
    {\cellcolor{red}mauvais}{Ne respecte pas ce critère, car pas de distinction entre une faible approbation ou une grande approbation.}%
    {\cellcolor{green}bien}{Cette méthode est faiblement manipulable, car elle est clair et chaque score est parfaitement indépendant.}

  \section{Jugement majoritaire}

  Chaque électeur donne une évaluation à chaque candidat.
  Chaque candidat reçoit une évaluation finale qui correspond à l'évaluation médiane qu'il a reçue (donc il y a au moins 
  50\% des électeurs qui ont donné cette évaluation ou plus).
  Le candidat élu est celui avec l'évaluation la plus élevée.
  Ce scrutin est décrit dans l'algorithme \aref{scrutin:jugement-maj}.

  L'évaluation peut être numérique, par exemple une note sur 20. Mais il n'est absolument pas sûr qu'une note de 12/20 ait
  la même signification pour tous les électeurs, ce qui pose problème pour interpréter les résultats. 
  Une autre technique consiste à donner des appréciations à chaque candidat (excellent, bien, passable …), l'avantage
  est que si les termes en français sont bien choisis, ils ont à peu près la même signification pour tous.

  Pour les exemples de ce document, ce sont des appréciations qui ont été choisies : 

  \begin{center}
  \colorbox{green}{Excellent} $\succ$
  \colorbox{green!50!yellow}{Bien} $\succ$
  \colorbox{green!25!yellow}{Correct} $\succ$
  \colorbox{yellow}{Passable} $\succ$
  \colorbox{orange}{Insufisant} $\succ$
  \colorbox{red}{A rejeter}
  \end{center}
  
  \begin{algorithm}[H]
  \caption{Scrutin au jugement majoritaire}%
  \label{scrutin:jugement-maj}
  \begin{algorithmic}[1]
    \REQUIRE{$votes$[n° du votant][n° de l'option] = Le jugement de l'option (5 = parfait; 0 = À rejeter)}
    \ENSURE{Liste des n° des choix vainqueurs}
    \STATE{$majorité \leftarrow \frac{\text{len}(votes)}{2}$}
  \FORALL{$vote \leftarrow votes$}
    \FORALL{$option \leftarrow vote$}
      \STATE{$jugement \leftarrow vote[option]$}
      \STATE{$total[option][jugement] ++$}
    \ENDFOR{}
    \ENDFOR{}
  % TODO
  \end{algorithmic}
  \end{algorithm}

  \begin{table}[H]
    \begin{center}
      \caption{Jugement majoritaire: manipulabilité du vote}%
      \label{fig:diff:jugmaj:caslim1}
      \adjustbox{valign=t}{
        \begin{subtable}[h]{0.45\textwidth}
          \centering
          \caption{Préférences des électeurs}
          \inputMake{_dyn/scrutin/jugmaj-caslim-1A.votes-n.tex}
        \end{subtable}
      }
      \adjustbox{valign=t}{
        \begin{subtable}[h]{0.45\textwidth}
          \centering
          \caption{Résultats}
          \inputMake{_dyn/scrutin/jugmaj-caslim-1A.jugmaj.tex}
        \end{subtable}
      }\\[1em]
      \adjustbox{valign=t}{
        \begin{subtable}[h]{0.45\textwidth}
          \centering
          \caption{Tentative de manipulation}
          \inputMake{_dyn/scrutin/jugmaj-caslim-1B.votes-n.tex}
        \end{subtable}
      }
      \adjustbox{valign=t}{
        \begin{subtable}[h]{0.45\textwidth}
          \centering
          \caption{Résultats}
          \inputMake{_dyn/scrutin/jugmaj-caslim-1B.jugmaj.tex}
        \end{subtable}
      }
    \end{center}
  \end{table}
  % TODO compresser ce graphique
  \subsection{Vote utile}

  Les tableaux \aref{fig:diff:jugmaj:caslim1} montrent une élection normale, puis la même élection avec une tentative
  de manipulation de la part de 20\% des électeurs. 
  Ces électeurs ont choisi de modifier leur vote en donnant une mauvaise jugement au candidat B qu'ils trouvaient 
  initialement correct.

  \subsection{Évaluation}
  \tabcritere%
    {\cellcolor{red}mauvais}{L'avis des 49\% de la population est ignorée, que ce soit la pire des évaluations ou l'évaluation médiane, ça ne change rien.}%
    {\cellcolor{green}bien}{Chaque candidat a une évaluation propre ne dépendant pas des autres candidats.}%
    {\cellcolor{red}mauvais}{Seul l'avis médian est pris en compte, on peut construire des simulations dans laquelle une majorité de personnes préférerait un même autre candidat au candidat élu.}%
%TODO ci-dessous REFORMULER éventuellement "mais n'est pas évident"     
    {\cellcolor{green!25!yellow}correct}{Le meilleure technique consiste à donner ses vraies préférences, cependant dans certain cas spécifiques, pour certains votants uniquement, une meilleure stratégie peut exister.}

  \section{Méthode de Coombs et vote alternatif}

  Chaque votant classe toutes ou partie des options dans l'ordre de leurs préférences.
  Puis on simule un scrutin à la majorité absolue sans limite de nombre de tours, en retirant le pire candidat
  après chaque tour.

  \subsection{Vote alternatif}
  Aussi appelé vote préférentiel, transférable ou Méthode de R, le vote alternatif retire à chaque 
  tour le candidat ayant été classé le moins souvent en première position.
  C'est donc une simulation exacte d'un scrutin à la majorité absolue sans limites de nombre de tours, 
  mais en retirant la possibilité de réflexion entre chaque tour. 
  Il a donc les mêmes désavantages que ce dernier (voir \aaref{diff:scrutin-majoritaire}).

  \subsection{Méthode de Coombs}
  La méthode de Coombs retire à chaque tour le candidat ayant été classé le plus souvent en dernière position.
  Cette méthode à l'avantage de prendre en compte les votes négatifs (les candidats classés en dernier).

  \subsection{Exemple}

  \begin{table}[h]
    \begin{center}
      \caption{Exemple de scrutin méthode de Coombs et vote alternatif}%
      \label{fig:diff:coombs:caslim1}
      \adjustbox{valign=t}{%
        \begin{subtable}[h]{0.45\textwidth}
          \centering
          \caption{Préférences des électeurs}%
          \label{fig:diff:coombs:caslim1:pref}
          \inputMake{_dyn/scrutin/maj1-caslim-1.votes.tex}
        \end{subtable}
        }\\[1em]
        \adjustbox{valign=t}{%
          \begin{subtable}[h]{0.45\textwidth}
            \centering
            \caption{Coombs}%
            \label{fig:diff:coombs:caslim1:coombs}
            \inputMake{_dyn/scrutin/maj1-caslim-1.coombs.tex}
          \end{subtable}
        }
        \adjustbox{valign=t}{%
          \begin{subtable}[h]{0.45\textwidth}
            \centering
            \caption{Vote alternatif}%
            \label{fig:diff:coombs:caslim1:alternatif}
            \inputMake{_dyn/scrutin/maj1-caslim-1.alternatif.tex}
          \end{subtable}
        }
    \end{center}
  \end{table}

  Les tableaux \aref{fig:diff:coombs:caslim1} montrent qu'il y a une réelle différence entre les 2 méthodes

  \subsection{Évaluation}
  \subsubsection{Méthode de Coombs}
  \tabcritere%
    {\cellcolor{orange}moyen}{En retirant le candidat le plus souvent en dernière position, on prend en compte l'avis des perdants. Cela ne suffit pas pour s'assurer qu'il n'y a pas de meilleur candidat dans le cas où un candidat est élu à 51\%.}%
    {\cellcolor{red}mauvais}{Les petits candidats «~volent~» des voix aux autres.}%
    {\cellcolor{red}mauvais}{Si les voix sont suffisamment divisées dans un des camps.}%
    {\cellcolor{orange}moyen}{Pose globalement les mêmes problèmes que le scrutin majoritaire (\aref{diff:scrutin-majoritaire}), mais est plus complexe à prédire, car il faut prédire tous les tours avant le vote. La meilleure technique de vote se base sur la prise en compte des sondages afin de choisir le vote le plus pertinent.}
  \subsubsection{Vote alternatif}
  \tabcritere%
    {\cellcolor{red}mauvais}{L'avis de ceux n'ayant pas choisi le vainqueur n'est pas pris en compte.}%
    {\cellcolor{red}mauvais}{Les petits candidats «~volent~» des voix aux autres.}%
    {\cellcolor{red}mauvais}{Si les voix sont suffisamment divisées dans un des camps.}%
    {\cellcolor{orange}moyen}{Pose globalement les mêmes problèmes que le scrutin majoritaire (\aref{diff:scrutin-majoritaire}), mais est plus complexe à prédire, car il faut prédire tous les tours avant le vote. La meilleure technique de vote se base sur la prise en compte des sondages afin de choisir le vote le plus pertinent.}


  
  %\begin{nota}[by me]{Ma note}\lipsum[1][1-5]\end{nota}
  %\begin{question}[by me]{Ma question}\lipsum[2][1-5]\end{question}
  %\begin{important}[by me]{Ma not imp!}\lipsum[3][1-5]\end{important}
  %\begin{warning}[by me]{My warn}\lipsum[4][1-5]\end{warning}

\end{document}
