\documentclass[../report]{subfiles}
\begin{document}
  \chapter{Introduction}

  \section{Définition}

  \paragraph{Vote électronique ou e-voting}
  Vote effectuer grâce à des outils informatique.
  Ce terme peut à la fois désigner un vote effectuer à distance via Internet ou un vote effectuer
  dans un bureau de vote à l'aide d'une machine à voter.
  Dans ce document nous nous focalisons sur le vote électronique à distance.
  Lorsque ce n'est pas préciser, nous parlons du vote électronique à distance et non pas dans un bureau de vote.

  \paragraph{Comptage électronique}
  \todo{}
  
  \paragraph{NIZK proof of knoledge}
  \todo{}

  \paragraph{Élection ou votation}
  Une élection est le fait qu'un groupe de personne (les électeurices) choississent une (ou plusieurs) personne
  pour les représenter ou pour toutes autres tâche.
  Une votation est le fait qu'un groupe de personne (les votant·es) choissient une (ou des) options.
  Une élection est donc simplement une votations ou les options possibles sont des personnes.
  Ce document ne précise pas à chaque fois que ce qui est dit pour l'un est également valable pour l'autre de 
  manière générale.

  \paragraph{Vote utile}\aref{diff:comp:util:util}
  \todo{}
  \paragraph{Quorum}
  \todo{}


  \section{Type de méthode de scrutin analyser}
  \todo{parler des scrutins exclusivements pour des élections vs permettant également de prendre des décisions}
  \section{candidat vs options}
  \section{Vote blanc/nuls/abstentions dans les simulations}
  \todo{Indiquer que les simulations faites dans ce document se base toutes sur une participation total, cepedant il y a la possibilité dans certain scrutin de refusé de classé certain candidat/option}
  \section{referendhum cas particulier d'une élection}
  \todo{un referendhum est juste une élection où les candidats sont "pour" et "contre"}
  \todo{Indiquer que l'on peut faire mieux}

  \section{Exemple}
  \todo{Certain exemple dans ce document sont issu de cas réel, ceci uniquement affin de montrer la plausibilité d'un tel cas, bien sûr le fonctionnement est identique en cas d'inversion des bords politiques des candidats/options dans les exemples}
  
  \todo{Montrer des exemples et expliquer les tableaux}

  \todo{Alice est la votante dans les exemples, présenté un exemple de structure utilisé dans 
  le document}

\end{document}

