\documentclass[report]{subfiles}
\begin{otherlanguage}{french}
  \titlesuffix{\publishablesummaryname}
\end{otherlanguage}
\begin{document}

    \begin{abstract}
    	
		Les scrutins sont omniprésents dans notre quotidien. Les élections et votations politiques sont celles auxquelles on pense le plus rapidement. 
		Les scrutins sont pourtant utilisés dans bien d'autres situations comme des décisions au sein d'associations ou d'entreprises.
		Dans de nombreux cas le scrutin utilisé est le scrutin majoritaire. 
		Ce scrutin a pourtant bien des défauts. Durant ce travail de Bachelor il a été comparé à d'autres scrutins (méthode de Condorcet, vote par approbation, méthode Borda, jugement majoritaire, ...) suite à cette comparaison la méthode Borda a été choisie pour être implémentée dans un système de e-voting.
		
		Ce proof-of-concept d'e-voting Borda, met en place la méthode Borda en essayant de garantir au maximum la confidentialité du vote. 
		S'il n'y a pas de compromission, seul le votant a la connaissance de son vote: ni le serveur central, ni ceux calculant les résultats n'ont accès 
		au contenu des bulletins de vote. L'implémentation se base sur la courbe écliptique Ristretto255.
		
    \end{abstract}


\end{document}

